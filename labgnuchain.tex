\labwork{Установка GNU toolchain}\label{gnuchaininstall}

\bigskip
Самая важная часть\ --- ставим GCC toolchain (набор инструментов)
для процессоров ARM, собранный для \file{\$TARGET = arm-none-eabi}.
Вариантов сборок для разработки для ARM под Windows много, есть и такие
дистрибутивы как \href{http://www.coocox.org/}{CooCox IDE}, включаеющие полный
комплект ПО одним пакетом. Ограничимся установкой варинта сборки под названием
Yagarto:

\bigskip\menu{\wcmd{\url{http://sourceforge.net/projects/yagarto/}}>Download}

\bigskip
Запускаем скачанный инсталлятор.

\bigskip\menu{Welcome>Next}

\menu{License>Accept>Next}

\menu{Choose Components>\checkbox\ Add YAGARTO to PATH>Next}

\menu{Destination folder>D:/ARM/Yaga>Next}

\menu{Start Menu Folder>YAGARTO>Install}

\menu{Installation Complete>Next>Finish}

\bigskip Яг\`{а} поставилась, теперь можно проверить что доступны базовые 
утилиты:

\bigskip Ассемблер
\begin{lstlisting}[style=con]
C:\Documents and Settings\pda>arm-none-eabi-as --version
GNU assembler (GNU Binutils) 2.23.1
Copyright 2012 Free Software Foundation, Inc.
This program is free software; you may redistribute it under the terms of
the GNU General Public License version 3 or later.
This program has absolutely no warranty.
This assembler was configured for a target of `arm-none-eabi'. 
\end{lstlisting}
\bigskip Линкер
\begin{lstlisting}[style=con]
C:\Documents and Settings\pda>arm-none-eabi-ld --version
GNU ld (GNU Binutils) 2.23.1
\end{lstlisting}
\bigskip Утилиты для работы с объектными файлами в формате ELF
\begin{lstlisting}[style=con]
C:\Documents and Settings\pda>arm-none-eabi-objdump --version
GNU objdump (GNU Binutils) 2.23.1
\end{lstlisting}
\begin{lstlisting}[style=con]
C:\Documents and Settings\pda>arm-none-eabi-objcopy --version
GNU objcopy (GNU Binutils) 2.23.1
\end{lstlisting}
\bigskip Препроцессор (\textbf{не} компилятор \cpp)
\begin{lstlisting}[style=con]
C:\Documents and Settings\pda>arm-none-eabi-cpp --version
arm-none-eabi-cpp (GCC) 4.7.2
Copyright (C) 2012 Free Software Foundation, Inc.
This is free software; see the source for copying conditions.  There is NO
warranty; not even for MERCHANTABILITY or FITNESS FOR A PARTICULAR PURPOSE.
\end{lstlisting}
\bigskip Компилятор Си
\begin{lstlisting}[style=con]
C:\Documents and Settings\pda>arm-none-eabi-gcc --version
arm-none-eabi-gcc (GCC) 4.7.2
\end{lstlisting}
\bigskip Компилятор \cpp
\begin{lstlisting}[style=con]
C:\Documents and Settings\pda>arm-none-eabi-g++ --version
arm-none-eabi-g++ (GCC) 4.7.2
\end{lstlisting}

\bigskip Утилита Make
\begin{lstlisting}[style=con]
C:\Documents and Settings\pda>make --version
"make" is not internal or external command.

C:\Documents and Settings\pda>arm-none-eabi-make --version
"make" is not internal or external command.

C:\Documents and Settings\pda>dir D:\ARM\Yaga\bin\*make*
 Volume D has no label.
 Serial #: 6588-9778

 Direcory contents D:\ARM\Yaga\bin

File not found
\end{lstlisting}

\bigskip Упс, а \make\ почему-то в комплект не включили \frownie.
Придется его ставить отдельно в \labref{gnuwinstall}.


