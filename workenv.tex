\chapter{Рабочая среда разработчика встраиваемых систем}

\labpart{Установка ПО}

\section{Выбор и установка операционной системы}

\subsection{MacOS}

Этот странный пока случай не рассматриваем\ --- у меня нет под рукой Мака \smiley.

\subsection{Windows}

Самый распространенный вариант. Вам придется ограничиться этим вариантом если вам не повезло
с поставщиком контроллера: windows-only ПО поддержки, например софт для программатора, или
внезапно библиотеки только для коммерческих компиляторов. Аналогичная ситуация будет в случае
покупки какого-нибудь специфичного оборудования (лог.анализатор, измерительное 
оборудование или просто принтер).

Установку ОС не рассматриваем.

\win{Секции текста книги, зависящие от ОС, будут выделены вот так.}

Для разработки встраиваемого ПО нужно поставить несколько пакетов, обеспечивающих
совместимость с UNIX средами \ref{winsoftinstall}.

\subsection{Linux}\label{linux}

\begin{enumerate}
\item \label{linuxref1}\linux\ удобен для разработчика, 
\item Если он вам не удобен, см п.\ref{linuxref1}
\end{enumerate}

\lin{Секции текста книги, зависящие от ОС, будут выделены вот так.}

\lin{\labwork{Установка Debian GNU/Linux}\label{debianinstall}}

\section{Установка инструментального программного обеспечения}

\win{\labwork{Установка инструментального ПО для Windows}\label{winsoftinstall}}

Создадим рабочий каталог, установим систему контроля версий \git\ref{git}\ и 
получим локальную копию проекта этой книги, содержащий кроме текста для издательской системы
\LaTeX\ еще и исходные коды библиотек, примеры кода и т.п., которые вы захотите
использовать в своих проектах.

%\labsection{Установка Git}\label{gitinstall}

\bigskip\wcmd{\url{http://git-scm.com/download/win}}

Запуститься закачка установочного пакета scm-git (\file{Git-1.9.4-preview20140611.exe}), после его загрузки
запустите установщик, 

\bigskip
\menu{Welcome>Next}

\bigskip
\menu{GNU GPL>Next} 

\bigskip
\menu{Select components>Windows Explorer Integration>Simple Context Menu>Git GUI here>Next}

\bigskip
\menu{Use Git and optional Unix tools from the Command Prompt>Next}

\bigskip
\menu{Use OpenSSH>Next}

\bigskip
\menu{Checkout Windows-style>Next}

\bigskip
\menu{Extracting files...}

\bigskip
\menu{Completing Setup>\uncheckbox\ View ReleaseNotes>Finish}

\bigskip
Проверим что \git\ правильно установился:

\bigskip\wcmd{cmd}

\bigskip
\begin{lstlisting}[style=con]
C:\Documents and Settings\pda>git --version
git version 1.9.4.msysgit.0
\end{lstlisting}

\bigskip
Первое, что вам следует сделать после установки \git а\ ---указать ваше имя и
адрес электронной почты. Это важно, потому что каждый коммит в \git е содержит
эту информацию, и она включена в коммиты, передаваемые вами:
\begin{lstlisting}[style=con]
C:\Documents and Settings\pda>git config --global user.name "Vasya Pupkin"
C:\Documents and Settings\pda>git config --global user.email no@mail.com
C:\Documents and Settings\pda>git config --global push.default simple
\end{lstlisting}

\bigskip
Эти настройки достаточно сделать только один раз, поскольку в этом случае 
\git\ будет использовать эти данные для всего, что вы делаете.
 Если для каких-то отдельных проектов вы хотите указать другое имя или
электронную почту, можно выполнить эту же команду без параметра \verb|--global|
в каталоге с нужным проектом.

\bigskip
Создаем каталог \directory{D:/ARM}\ и выгружаем текущую копию этой книги из репозитория
\url{https://github.com/ponyatov/CortexMx}

\bigskip\wcmd{cmd}

\bigskip
\begin{lstlisting}[style=con]
C:\Documents and Settings\ponyatov>D:
D:\>mkdir \ARM
D:\>cd \ARM
D:\ARM>git clone --depth=1 https://github.com/ponyatov/CortexMx.git book
\end{lstlisting}



\lin{\labwork{Установка инструментального ПО для \linux}\label{linsoftinstall}}

\chapter{Управление версиями при написании ПО}

\section{Git}\label{git}\cite{progit}
