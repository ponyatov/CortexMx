\chapter{Рабочая среда разработчика встраиваемых систем}

\labpart{Установка ПО}

\section{Выбор и установка операционной системы}

\subsection{MacOS}

Этот странный пока случай не рассматриваем\ --- у меня нет под рукой Мака \smiley.

\subsection{Windows}

Самый распространенный вариант. Вам придется ограничиться этим вариантом если вам не повезло
с поставщиком контроллера: windows-only ПО поддержки, например софт для программатора, или
внезапно библиотеки только для коммерческих компиляторов. Аналогичная ситуация будет в случае
покупки какого-нибудь специфичного оборудования (лог.анализатор, измерительное 
оборудование или просто принтер).

Установку ОС не рассматриваем.

\win{Секции текста книги, зависящие от ОС, будут выделены вот так.}

Для разработки встраиваемого ПО нужно поставить несколько пакетов, обеспечивающих
совместимость с UNIX средами \ref{winsoftinstall}.

\subsection{Linux}\label{linux}

\begin{enumerate}
\item \label{linuxref1}\linux\ удобен для разработчика, 
\item Если он вам не удобен, см п.\ref{linuxref1}
\end{enumerate}

\lin{Секции текста книги, зависящие от ОС, будут выделены вот так.}

\lin{\labwork{Установка Debian GNU/Linux}\label{debianinstall}}

\section{Установка инструментального программного обеспечения}

\win{\labwork{Установка инструментального ПО для Windows}\label{winsoftinstall}}

Создадим рабочий каталог, установим систему контроля версий \git\ref{git}\ и 
получим локальную копию проекта этой книги, содержащий кроме текста для издательской системы
\LaTeX\ еще и исходные коды библиотек, примеры кода и т.п., которые вы захотите
использовать в своих проектах:

\bigskip\wcmd{\url{http://git-scm.com/download/win}}

Запуститься закачка установочного пакета scm-git (\file{Git-1.9.4-preview20140611.exe}), после его загрузки
запустите установщик, 

\bigskip
\menu{Welcome>Next}

\bigskip
\menu{GNU GPL>Next} 

\bigskip
\menu{Select components>Windows Explorer Integration>Simple Context Menu>Git GUI here>Next}

\bigskip
\menu{Use Git and optional Unix tools from the Command Prompt>Next}

\bigskip
\menu{Use OpenSSH>Next}

\bigskip
\menu{Checkout Windows-style>Next}

\bigskip
\menu{Extracting files...}

\bigskip
\menu{Completing Setup>\uncheckbox\ View ReleaseNotes>Finish}

\bigskip
Проверим что \git\ правильно установился:

\bigskip\wcmd{cmd}

\bigskip
\begin{lstlisting}[style=con]
C:\Documents and Settings\pda>git --version
git version 1.9.4.msysgit.0
\end{lstlisting}

\bigskip
Первое, что вам следует сделать после установки \git а\ ---указать ваше имя и
адрес электронной почты. Это важно, потому что каждый коммит в \git е содержит
эту информацию, и она включена в коммиты, передаваемые вами:
\begin{lstlisting}[style=con]
C:\Documents and Settings\pda>git config --global user.name "Vasya Pupkin"
C:\Documents and Settings\pda>git config --global user.email no@mail.com
C:\Documents and Settings\pda>git config --global push.default simple
\end{lstlisting}

\bigskip
Эти настройки достаточно сделать только один раз, поскольку в этом случае 
\git\ будет использовать эти данные для всего, что вы делаете.
 Если для каких-то отдельных проектов вы хотите указать другое имя или
электронную почту, можно выполнить эту же команду без параметра \verb|--global|
в каталоге с нужным проектом.

\bigskip
Создаем каталог \directory{D:/ARM}\ и выгружаем текущую копию этой книги из репозитория
\url{https://github.com/ponyatov/CortexMx}

\bigskip\wcmd{cmd}

\bigskip
\begin{lstlisting}[style=con]
C:\Documents and Settings\ponyatov>D:
D:\>mkdir \ARM
D:\>cd \ARM
D:\ARM>git clone --depth=1 https://github.com/ponyatov/CortexMx.git book
\end{lstlisting}

\bigskip
Самая важная часть\ --- ставим GCC toolchain (набор инструментов)
для процессоров ARM, собранный для \file{\$TARGET = arm-none-eabi}.
Вариантов сборок для разработки для ARM под Windows много, есть и такие
дистрибутивы как \href{http://www.coocox.org/}{CooCox IDE}, включаеющие полный
комплект ПО одним пакетом. Ограничимся установкой варинта сборки под названием
Yagarto:

\bigskip\menu{\wcmd{\url{http://sourceforge.net/projects/yagarto/}}>Download}

\bigskip
Запускаем скачанный инсталлятор.

\bigskip\menu{Welcome>Next}

\bigskip\menu{License>Accept>Next}

\bigskip\menu{Choose Components>\checkbox\ Add YAGARTO to PATH>Next}

\bigskip\menu{Destination folder>D:/ARM/Yaga>Next}

\bigskip\menu{Start Menu Folder>YAGARTO>Install}

\bigskip\menu{Installation Complete>Next>Finish}

\bigskip Яг\`{а} поставилась, теперь можно проверить что доступны базовые 
утилиты:

\bigskip Ассемблер
\begin{lstlisting}[style=con]
C:\Documents and Settings\pda>arm-none-eabi-as --version
GNU assembler (GNU Binutils) 2.23.1
Copyright 2012 Free Software Foundation, Inc.
This program is free software; you may redistribute it under the terms of
the GNU General Public License version 3 or later.
This program has absolutely no warranty.
This assembler was configured for a target of `arm-none-eabi'. 
\end{lstlisting}
\bigskip Линкер
\begin{lstlisting}[style=con]
C:\Documents and Settings\pda>arm-none-eabi-ld --version
GNU ld (GNU Binutils) 2.23.1
\end{lstlisting}
\bigskip Утилиты для работы с объектными файлами в формате ELF
\begin{lstlisting}[style=con]
C:\Documents and Settings\pda>arm-none-eabi-objdump --version
GNU objdump (GNU Binutils) 2.23.1
\end{lstlisting}
\begin{lstlisting}[style=con]
C:\Documents and Settings\pda>arm-none-eabi-objcopy --version
GNU objcopy (GNU Binutils) 2.23.1
\end{lstlisting}
\bigskip Препроцессор (\textbf{не} компилятор \cpp)
\begin{lstlisting}[style=con]
C:\Documents and Settings\pda>arm-none-eabi-cpp --version
arm-none-eabi-cpp (GCC) 4.7.2
Copyright (C) 2012 Free Software Foundation, Inc.
This is free software; see the source for copying conditions.  There is NO
warranty; not even for MERCHANTABILITY or FITNESS FOR A PARTICULAR PURPOSE.
\end{lstlisting}
\bigskip Компилятор Си
\begin{lstlisting}[style=con]
C:\Documents and Settings\pda>arm-none-eabi-gcc --version
arm-none-eabi-gcc (GCC) 4.7.2
\end{lstlisting}
\bigskip Компилятор \cpp
\begin{lstlisting}[style=con]
C:\Documents and Settings\pda>arm-none-eabi-g++ --version
arm-none-eabi-g++ (GCC) 4.7.2
\end{lstlisting}

\bigskip Утилита Make
\begin{lstlisting}[style=con]
C:\Documents and Settings\pda>make --version
"make" is not internal or external command.

C:\Documents and Settings\pda>arm-none-eabi-make --version
"make" is not internal or external command.

C:\Documents and Settings\pda>dir D:\ARM\Yaga\bin\*make*
 Volume D has no label.
 Serial #: 6588-9778

 Direcory contents D:\ARM\Yaga\bin

File not found
\end{lstlisting}

\bigskip Упс, а make почему-то в комплект не включили \frownie

\bigskip Для совместимости скриптов придется поставить несколько пакетов
из \file{GnuWin32}:

\bigskip\wcmd{\url{http://gnuwin32.sourceforge.net/packages.html}}

\bigskip\menu{\keys{Ctrl+F}>coreutils>\keys{Ctrl+Setup}}

\menu{\keys{Ctrl+F}>wget>\keys{Ctrl+Setup}}

\menu{\keys{Ctrl+F}>gnu make>\keys{Ctrl+Setup}}

\bigskip\file{coreutils-5.3.0.exe} основные UNIX-утилиты типа rm ls , собранные
под win32

\bigskip\menu{Welcome>Next}

\bigskip\menu{License>Accept>Next}

\bigskip\menu{Folder>D:/ARM/GnuWin32>Next}

\bigskip\menu{Components>Next}

\bigskip\menu{Start Menu>GnuWin32/CoreUtils>Next}

\bigskip\menu{Select Additional>Next}

\bigskip\menu{Ready to Install>Next}

\bigskip\menu{Compliting>Finish}

\bigskip Аналогично ставим:

\file{make-3.81.exe} утилита make

\file{wget-1.11.4.exe} консольная утилита загрузки файлов по HTTP/FTP

\file{grep-2.5.4.exe} утилита поиска строк в файлах и stdin/stdout потоке

\bigskip Чтобы утилиты \file{GnuWin32}\ были доступны, нужно прописать
переменную пользователя \verb|$PATH|\ в системном окружении.

\bigskip\menu{Пуск>Настройка>Панель управления>Система>Дополнительно>Переменные
среды}

\bigskip\menu{Переменные среды>переменные пользователя>Создать/Изменить}

\bigskip\menu{Имя переменной>PATH}

\menu{Значение переменной>добавить в начало
D:/ARM/GnuWin32/bin;D:/ARM/Yaga/bin;..}

\menu{Ok>Ok>Ok}

\bigskip Проверяем:
\begin{lstlisting}[style=con]
C:\Documents and Settings\pda>ls -la
total 3111
drwxr-xr-x   29 pda      user        0 Jul  4 14:03 .
drwxr-xr-x    9 pda      user        0 Oct  8  2013 ..
-rw-r--r--    1 pda      user     5242 May 22 14:29 .bash_history
drwxr-xr-x    2 pda      user        0 May 23  2013 .borland
drwxr-xr-x   18 pda      user        0 Sep  4  2013 .ccache
drwxr-xr-x    3 pda      user        0 Mar 26  2013 .eclipse
\end{lstlisting}
\begin{lstlisting}[style=con]
C:\Documents and Settings\pda>wget --version
GNU Wget 1.7

Copyright (C) 1995, 1996, 1997, 1998, 2000, 2001 Free Software Foundation, Inc.
This program is distributed in the hope that it will be useful,
but WITHOUT ANY WARRANTY; without even the implied warranty of
MERCHANTABILITY or FITNESS FOR A PARTICULAR PURPOSE.  See the
GNU General Public License for more details.

Originally written by Hrvoje Niksic <hniksic@arsdigita.com>.
\end{lstlisting}
\begin{lstlisting}[style=con]
C:\Documents and Settings\pda>make --version
GNU Make 3.81
Copyright (C) 2006  Free Software Foundation, Inc.
This is free software; see the source for copying conditions.
There is NO warranty; not even for MERCHANTABILITY or FITNESS FOR A
PARTICULAR PURPOSE.

This program built for i386-pc-mingw32
\end{lstlisting}

\lin{\labwork{Установка инструментального ПО для \linux}\label{linsoftinstall}}

\chapter{Управление версиями при написании ПО}

\section{Git}\label{git}\cite{progit}
