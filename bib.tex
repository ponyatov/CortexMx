\addcontentsline{toc}{part}{Литература}
\begin{thebibliography}{9}

\bibitem{armbuka}{\url{https://github.com/ponyatov/CortexMx} \thetitle}

\bibitem{leaps}{\copyright\ Quantum Leaps}

\bibitem{milandr}{\url{http://milandr.ru/} ЗАО <<ПКК Миландр>>}

\bibitem{progit}{\url{http://git-scm.com/book/ru} перевод:
Scott Chacon
\textbf{Pro Git}
}

\bibitem{habraQP}{\url{http://habrahabr.ru/post/114239/} хабра: Quantum Leaps QP
и диаграммы состояний в UML}

\bibitem{quantumleaps}{\url{http://www.state-machine.com/}
Quantum$^{\circledR}L^{e}aPs$ State Machines \& Tools}

\bibitem{mks1}{\url{http://makesystem.net/?p=988} Изучаем ARM. Собираем свою IDE
для ARM}

\bibitem{mks2}{\url{http://makesystem.net/?p=2146} Изучаем ARM. Отладка ARM
приложений в Eclipse IDE}

\bibitem{lvovsky}{Львовский С.М. Набор и вёрстка в пакете \LaTeX}

\end{thebibliography}
