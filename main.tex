\documentclass[oneside,14pt]{book}

\usepackage[T1,T2A]{fontenc}
\usepackage[utf8]{inputenc}
\usepackage[english,russian]{babel}
\usepackage{indentfirst}

%\usepackage[paperwidth=15cm,paperheight=7.5cm]{geometry} % планшет
%\usepackage[paperwidth=297mm,paperheight=210mm]{geometry} % A4
%\usepackage[paperwidth=148mm,paperheight=105mm]{geometry} % A5
\usepackage[paperwidth=18cm,paperheight=13cm,margin=5mm]{geometry} % экран*2

%\usepackage[colorlinks=true,
%]{hyperref}
%\newcommand{\email}[2]{#1\ \href{mailto:#2}{<\nolinkurl{#2}>}}
% %\newcommand{\email}[2]{\emph{#1\ <#2>}}
\usepackage[unicode,colorlinks,
pdftitle={Azbuka ARMaturschika (ru)},
pdfauthor={(c) Dmitry Ponyatov <dponyatov@gmail.com>, SSAU ASCL},
pdfsubject={ru manual on writing programs for Cortex-M MCUs},
pdfkeywords={ARM} {Cortex} {MCU} {ARMatura} {Arduino} {SSAU}
]{hyperref}

\usepackage{wrapfig}
\usepackage{graphicx}
\usepackage{epstopdf}
\DeclareGraphicsExtensions{.eps}

\usepackage{listings}
\usepackage{dirtree}
\usepackage[usenames,dvipsnames,svgnames]{xcolor}
\newcommand{\cppcolor}{\color[rgb]{0.94, 0.97, 1.0}} % Alice blue
\newcommand{\asmcolor}{\color[rgb]{0.98, 0.92, 0.84}} % Antique white
\newcommand{\concolor}{\color[rgb]{0.88, 1.0, 1.0}} % Light cyan

%\definecolor{cppcolor}{rgb}{0.94, 0.97, 1.0}
\lstset{frame=single,
numbers=left, numberstyle=\small,
commentstyle=\color{cyan}\texttt,
tabsize=4,
keywordstyle=\color{blue}\texttt
}
\usepackage{lstlangarm}

\lstdefinestyle{cpp}{language=C++,backgroundcolor=\cppcolor}
\lstdefinestyle{asm}{language={[ARM]Assembler},backgroundcolor=\asmcolor}
\lstdefinestyle{con}{backgroundcolor=\concolor}

\newcommand{\cm}[1]{Cortex-M#1}
\newcommand{\cx}{\cm{x}}

\newcommand{\vld}{STM32VLDISCOVERY}

%\renewcommand{\url}[1]{\textbf{#1}}
\newcommand{\email}[1]{$<$\href{mailto:#1}{\textbf{#1}}$>$}

\newcommand{\cpp}{$C^{+^{+}}$}

\newcommand{\cp}[1]{\footnote{копипаста: #1}}

\newcommand{\thmod}{Thumb}
\newcommand{\armod}{ARM}

\newcommand{\Reg}[1]{\textbf{#1}}
\newcommand{\R}[1]{\Reg{R#1}}

\newcommand{\periph}[1]{\texttt{#1}}
\newcommand{\jtag}{\periph{JTAG}}

\usepackage{wasysym} % smileys
\usepackage{gensymb} % celsius
\usepackage{amssymb} % windows key
\usepackage{textcomp} % bigcircle

\usepackage[os=win]{menukeys}
\newcommand{\winstart}{$\boxplus$}
\newcommand{\file}[1]{\textbf{\textsf{#1}}}
\newcommand{\window}[1]{\textbf{\textit{#1}}}
\newcommand{\alarm}[1]{{\color{DarkRed}#1}}
\newcommand{\wcmd}[1]{\keys{\winstart+R}\ \directory{#1}}
\newcommand{\checkbox}{$\boxtimes$}
\newcommand{\uncheckbox}{$\square$}

\newcommand{\win}[1]{\includegraphics[height=10ex]{fig/winlogo.jpg} #1}
\newcommand{\lin}[1]{\includegraphics[height=10ex]{fig/linuxcolor.png} #1}

\newcommand{\linux}{Linux}
\newcommand{\git}{Git}
\newcommand{\eclipse}{\textcircled{$\equiv$}\textsc{eclipse}}
%\newcommand{\term}[1]{\underline{#1}}
\newcommand{\term}[1]{\underline{\color{DarkBlue} #1}}

\usepackage{tocloft}
\newcommand{\listlabname}{Лабораторные работы}
\newlistof{lab}{ex}{\listlabname}
\newcommand{\labpart}[1]{\addcontentsline{ex}{part}{#1}}
\newcounter{labworkcounter}
\newcommand{\labwork}[1]{
\refstepcounter{labworkcounter}
\section*{ЛР\thelabworkcounter: #1}
\addcontentsline{toc}{subsection}{ЛР\thelabworkcounter: #1}
\addcontentsline{ex}{section}{ЛР\thelabworkcounter: #1}
}
\newcommand{\labref}[1]{ЛР\ref{#1}}

\newcommand{\mytitle}[1]{
\title{\Huge{Азбука халтурщика-ARMатурщика}\\
#1\\
\normalsize{учебный курс по микроконтроллерам \cx:\\
Миландр 1986ВЕ, STM32F, LPC21xx}}
}

\author{\copyright\\
Понятов Д.А. \email{dponyatov@gmail.com}, ИКП СГАУ, \\
Недяк С.П. \email{fvs@fet.tusur.ru}, ТУСУР
}


\begin{document}

\mytitle{\ }
\maketitle
\tableofcontents
\listoflab

\part{Обзор семейства микроконтроллеров \cx}

\chapter{Архитектура ARM}

\cp{\url{http://ru.wikipedia.org/wiki/ARM}} %\_(архитектура)

Архитектура ARM (Advanced RISC Machine, Acorn RISC Machine, 
усовершенствованная RISC-машина)\ --- семейство лицензируемых 32-битных и 
64-битных микропроцессорных ядер разработки компании ARM Limited.

Среди лицензиатов: практически все заметные разработчики цифровых
электронных компонентов.
Многие лицензиаты делают собственные версии ядер на базе ARM.

Значимые семейства процессоров: ARM7, ARM9, ARM11 и Cortex.

В 2007 году около 98\% из более чем миллиарда мобильных телефонов, продаваемых 
ежегодно, были оснащены по крайней мере одним процессором ARM. По состоянию 
на 2009 на процессоры ARM приходилось до 90\% всех встроенных 32-разрядных 
процессоров. Процессоры ARM широко используются в потребительской 
электронике\ --- в том числе КПК, мобильных телефонах, цифровых носителях и 
плеерах, портативных игровых консолях, калькуляторах и компьютерных п
ериферийных устройствах, таких как жесткие диски или маршрутизаторы.

Эти процессоры имеют низкое энергопотребление, поэтому находят широкое 
применение во встраиваемых системах и преобладают на рынке мобильных 
устройств, для которых данный фактор немаловажен.

В настоящее время значимыми являются несколько семейств процессоров ARM:

\begin{itemize}
\item ARM7 (с тактовой частотой до 60-72 МГц), предназначенные, например, для 
недорогих мобильных телефонов и встраиваемых решений средней производительности. 
В настоящее время активно вытесняется новым семейством Cortex.
\item ARM9, ARM11 (с частотами до 1 ГГц) для продвинутых телефонов, карманных 
компьютеров и встраиваемых решений высокой производительности.
\item Cortex A\ --- новое семейство процессоров на смену ARM9 и ARM11.
\item Cortex M\ --- новое семейство процессоров на смену ARM7, также 
призванное занять новую для ARM нишу встраиваемых решений низкой 
производительности. В семействе присутствуют три значимых ядра: \cm{0}, 
\cm{3} и \cm{4}.
\end{itemize}

\paragraph{Архитектура}

Существует спецификация архитектуры ARM Cortex, которая разграничивает все типы опций, 
которые поддерживает ARM, так как детали реализации каждого типа процессора м
огут отличаться. Архитектура развивалась с течением времени, и начиная с 
ARMv7 были определены 3 профиля:
\begin{itemize}
\item{A (application)} для устройств, требующих высокой производительности (смартфоны, планшеты)
\item{R (real time)} для приложений, работающих в реальном времени,
\item{M (microcontroller)} для микроконтроллеров и недорогих встраиваемых устройств.
\end{itemize}

\paragraph{Режимы процессора}

Процессор может находиться в одном из следующих рабочих режимов:
\begin{itemize}
\item{User mode} — обычный режим выполнения программ. В этом режиме 
выполняется большинство программ.
\item{Fast Interrupt (FIQ)} — режим быстрого прерывания (меньшее время с
рабатывания)
\item{Interrupt (IRQ)} — основной режим прерывания.
\item{System mode} — защищённый режим для использования операционной системой.
\item{Abort mode} — режим, в который процессор переходит при возникновении 
ошибки доступа к памяти (доступ к данным или к инструкции на этапе 
prefetch конвейера).
\item{Supervisor mode} — привилегированный пользовательский режим.
\item{Undefined mode} — режим, в который процессор входит при попытке 
выполнить неизвестную ему инструкцию.
\end{itemize}

Переключение режима процессора происходит при возникновении соответствующего 
исключения, или же модификацией регистра статуса.

\paragraph{Набор команд ARM}

Режим, в котором исполняется 32-битный набор команд.

\paragraph{Набор команд \thmod}

Для улучшения плотности кода процессоры, начиная с ARM7TDMI, снабжены режимом 
\thmod. В этом режиме процессор выполняет альтернативный набор 16-битных 
команд. Большинство из этих 16-разрядных команд переводятся в нормальные 
команды ARM. Уменьшение длины команды достигается за счет сокрытия некоторых 
операндов и ограничения возможностей адресации по сравнению с режимом полного 
набора команд ARM.

В режиме \thmod\ меньшие коды операций обладают меньшей функциональностью. 
Например, только ветвления могут быть условными, и многие коды операций имеют 
ограничение на доступ только к половине главных регистров процессора. Более 
короткие коды операций в целом дают большую плотность кода, хотя некоторые 
операции требуют дополнительных команд. В ситуациях, когда порт памяти или 
ширина шины ограничены 16 битами, более короткие коды операций режима 
\thmod\ становятся гораздо производительнее по сравнению с обычным 32-битным ARM 
кодом, так как меньший программный код придется загружать в процессор при 
ограниченной пропускной способности памяти.

Аппаратные средства типа Game Boy Advance, как правило, имеют небольшой объём 
оперативной памяти доступной с полным 32-битным информационным каналом. Но 
большинство операций выполняется через 16-битный или более узкий информационный 
канал. В этом случае имеет смысл использовать \thmod\ код и вручную 
оптимизировать некоторые тяжелые участки кода, используя переключение в 
режим \armod.

\paragraph{Набор команд \thmod2}

\thmod2 — технология, стартовавшая с ARM1156 core, анонсированного в 2003 
году. Он расширяет ограниченный 16-битный набор команд Thumb дополнительными 
32-битными командами, чтобы задать набору команд дополнительную ширину. Цель 
\thmod2\ --- достичь плотности кода как у Thumb, и производительности как у 
набора команд \armod\ на 32 битах. Можно сказать, что в ARMv7 эта цель была 
достигнута.

\thmod2 расширяет как команды \armod, так и команды \thmod\ ещё большим 
количеством команд, включая управление битовым полем, табличное ветвление, 
условное исполнение. Новый язык «Unified Assembly Language» (UAL) поддерживает 
создание команд как для ARM, так и для Thumb из одного и того же исходного 
кода. Версии \thmod\ на ARMv7 выглядят как код ARM. Это требует осторожности и 
использования новой команды if-then, которая поддерживает исполнение до 4 
последовательных команд испытываемого состояния. Во время компиляции в ARM 
код она игнорируется, но во время компиляции в код \thmod2 генерирует команды.

\paragraph{Набор команд Jazelle}

Jazelle — это технология, которая позволяет байткоду Java исполняться прямо 
в архитектуре ARM в качестве 3-го состояния исполнения (и набора команд) 
наряду с обычными командами ARM и режимом Thumb. Поддержка технологии Jazelle 
обозначается буквой «J» в названии процессора — например, ARMv5TEJ. Данная 
технология поддерживается начиная с архитектуры ARMv6, хотя новые ядра 
содержат лишь ограниченные реализации, которые не поддерживают аппаратного 
ускорения.

\paragraph{ARMv8 и набор команд ARM 64 бит}

В конце 2011 года была опубликована новая версия архитектуры, ARMv8. В ней 
появилось определение архитектуры AArch64, в которой исполняется 64-битный 
набор команд A64. Поддержка 32-битных команд получила название A32 и
 исполняется на архитектурах AArch32. Инструкции Thumb поддерживаются в 
 режиме T32, только при использовании 32-битных архитектур. Допускается 
 исполнение 32-битных приложений в 64-битной ОС, и запуск виртуализованной 
 32-битной ОС при помощи 64-битного гипервизора.[47] Applied Micro, AMD, 
 Broadcom, Calxeda, HiSilicon, Samsung, STM и другие заявили о планах по 
 использованию ARMv8. Ядра Cortex-A53 и Cortex-A57, поддерживающие ARMv8, 
 были представлены компанией ARM 30 октября 2012 года.[48]

Как AArch32, так и AArch64, поддерживают VFPv3, VFPv4 и advanced SIMD (NEON). 
Также добавлены криптографические инструкции для работы с AES, SHA-1 и SHA-256.

\paragraph{Условное исполнение}

Одним из существенных отличий архитектуры ARM от других архитектур ЦПУ 
является так называемая предикация — возможность условного исполнения команд. 
Под «условным исполнением» здесь понимается то, что команда будет выполнена 
или проигнорирована в зависимости от текущего состояния флагов состояния 
процессора.

В то время как для других архитектур таким свойством, как правило, обладают 
только команды условных переходов, в архитектуру ARM была заложена 
возможность условного исполнения практически любой команды. Это было 
достигнуто добавлением в коды их инструкций особого 4-битового поля 
(предиката). Одно из его значений зарезервировано на то, что инструкция 
должна быть выполнена безусловно, а остальные кодируют то или иное сочетание 
условий (флагов). С одной стороны, с учётом ограниченности общей длины 
инструкции, это сократило число бит, доступных для кодирования смещения в 
командах обращения к памяти, но с другой — позволило избавляться от 
инструкций ветвления при генерации кода для небольших if-блоков.

Пример, обычно рассматриваемый для иллюстрации — основанный на вычитании 
алгоритм Евклида. В языке C он выглядит так:

\begin{lstlisting}[style=cpp,title={алгоритм Евклида}]
while (i != j) {
       if (i > j)
           i -= j;
       else
           j -= i;
}
\end{lstlisting}

А на ассемблере ARM — так:

\begin{lstlisting}[style=asm]
loop CMP Ri, Rj; set condition "NE" if (i != j),
						; "GT" if (i > j),
						; or "LT" if (i < j)
	SUBGT  Ri, Ri, Rj   ; if "GT" (greater than), i = i-j;
	SUBLT  Rj, Rj, Ri   ; if "LT" (less than), j = j-i;
	BNE    loop         ; if "NE" (not equal), then loop
\end{lstlisting}

Из кода видно, что использование предикации позволило полностью избежать 
ветвления в операторах else и then. Заметим, что если Ri и Rj равны, то ни 
одна из SUB инструкций не будет выполнена, полностью убирая необходимость в 
ветке, реализующей проверку while при каждом начале цикла, что могло быть 
реализовано, например, при помощи инструкции SUBLE (меньше либо равно).

Один из способов, которым уплотнённый (Thumb) код достигает большей экономии 
объёма — это именно удаление 4-битового предиката из всех инструкций, кроме 
ветвлений.

\paragraph{Другие особенности}

Другая особенность набора команд это возможность соединять сдвиги и вращения 
в инструкции «обработки информации» (арифметическую, логическую, движение 
регистр-регистр) так, что, например выражение С:
\begin{lstlisting}[style=cpp]
a += (j << 2);
\end{lstlisting}

может быть преобразовано в команду из одного слова и одного цикла в ARM:

\begin{lstlisting}[style=asm]
ADD Ra, Ra, Rj, LSL #2
\end{lstlisting}

Это приводит к тому, что типичные программы ARM становятся плотнее, чем 
обычно, с меньшим доступом к памяти. Таким образом, конвейер используется 
гораздо более эффективно. Даже несмотря на то, что ARM работает на скоростях, 
которые многие бы сочли низкими, он довольно-таки легко конкурирует с 
многими более сложными архитектурами ЦПУ.

ARM процессор также имеет некоторые особенности, редко встречающиеся в 
других архитектурах RISC — такие, как адресация относительно счетчика 
команд (на самом деле счетчик команд ARM является одним из 16 регистров), 
а также пре- и пост-инкрементные режимы адресации.

Другая особенность, которую стоит отметить, это то, что некоторые ранние 
ARM процессоры (до ARM7TDMI), например, не имеют команд для хранения 
2-байтных чисел. Таким образом, строго говоря, для них невозможно 
сгенерировать эффективный код, который бы вел себя так, как ожидается от 
объектов С, типа \verb|volatile int16_t|.

\paragraph{Сопроцессоры}

Архитектура предоставляет способ расширения набора команд, используя 
сопроцессоры, которые могут быть адресованы, используя MCR, MRC, MRRC, 
MCRR и похожие команды. Пространство сопроцессора логически разбито на 16 
сопроцессоров с номерами от 0 до 15, причем 15-й зарезервирован для 
некоторых типичных функций управления, типа управления кэш-памятью и 
операции блока управления памятью (на процессорах, в которых они есть).

В машинах на основе ARM периферийные устройства обычно подсоединяются к 
процессору путем сопоставления их физических регистров в памяти ARM или в 
памяти сопроцессора, или путем присоединения к шинам, которые в свою очередь 
подсоединяются к процессору. Доступ к сопроцессорам имеет большее время 
ожидания, поэтому некоторые периферийные устройства проектируются для 
доступа в обоих направлениях. В остальных случаях разработчики чипов лишь 
пользуются механизмом интеграции сопроцессора. Например, движок обработки 
изображений должен состоять из малого ядра ARM7TDMI, совмещенного с 
сопроцессором, который поддерживает примитивные операции по обработке 
элементарных кодировок HDTV.

\paragraph{Усовершенствованный SIMD (NEON)}

Расширение усовершенствованного SIMD, также называемое технологией NEON — это 
комбинированный 64- и 128-битный набор команд SIMD (single instruction 
multiple data), который обеспечивает стандартизованное ускорение для медиа 
приложений и приложений обработки сигнала. NEON может выполнять декодирование 
аудио формата mp3 на частоте процессора в 10 МГц, и может работать с речевым 
кодеком GSM AMR (adaptive multi-rate) на частоте более 13МГц. Он обладает
 внушительным набором команд, отдельными регистровыми файлами, и независимой 
 системой исполнения на аппаратном уровне. NEON поддерживает 8-, 16-, 32-, 
 64-битную информацию целого типа, одинарной точности и с плавающей запятой, 
 и работает в операциях SIMD по обработке аудио и видео (графика и игры).
  В NEON SIMD поддерживает до 16 операций единовременно.
VFP

\paragraph{Технология VFP}

 (Vector Floating Point, вектора чисел с плавающей запятой) — расширение 
 сопроцессора в архитектуре ARM. Она производит низкозатратные вычисления 
 над числами с плавающей запятой одинарной/двойной точности, в полной мере
  соответствующие стандарту ANSI/IEEE Std 754—1985 Standard for Binary
  Floating-Point Arithmetic. VFP производит вычисления с плавающей запятой, 
  подходящие для широкого спектра приложений — например, для КПК, смартфонов, 
  сжатие звука, трёхмерной графики и цифрового звука, а также принтеров и 
  телеприставок. Архитектура VFP также поддерживает исполнение коротких 
  векторных команд. Но, поскольку процессор выполняет операции 
  последовательно над каждым элементом вектора, то VFP нельзя назвать 
  истинным SIMD набором инструкций. Этот режим может быть полезен в графике 
  и приложениях обработки сигнала, так как он позволяет уменьшить размер кода 
  и выработку команд.

Другие сопроцессоры с плавающей запятой и/или SIMD, находящиеся в ARM 
процессорах включают в себя FPA, FPE, iwMMXt. Они обеспечивают ту же 
функциональность, что и VFP, но не совместимы с ним на уровне опкодов.

\paragraph{Отладка}

Все современные процессоры ARM включают аппаратные средства отладки, так как 
без них отладчики ПО не смогли бы выполнить самые базовые операции типа 
остановки, отступа, установка контрольных точек после перезагрузки.

Архитектура ARMv7 определяет базовые средства отладки на архитектурном 
уровне. К ним относятся точки останова, точки просмотра и выполнение команд 
в режиме отладки. Такие средства были также доступны с модулем отладки 
EmbeddedICE. Поддерживаются оба режима — остановки и обзора. Реальный 
транспортный механизм, который используется для доступа к средствам отладки, 
не специфицирован архитектурно, но реализация, как правило, включает 
поддержку \jtag.

Существует отдельная архитектура отладки «с обзором ядра», которая не 
требуется архитектурно процессорами ARMv7.

\paragraph{Регистры}

ARM предоставляет 31 регистр общего назначения разрядностью 32 бит. 
В зависимости от режима и состояния процессора пользователь имеет доступ 
только к строго определённому набору регистров. В ARM state разработчику 
постоянно доступны 17 регистров:

\begin{itemize}
\item 13 регистров общего назначения (\R{0}..\R{12}).
\item Stack Pointer (\R{13}) — содержит указатель стека выполняемой программы.
\item Link register (\R{14}) — содержит адрес возврата в инструкциях ветвления.
\item Program Counter (\R{15}) — биты [31:1] содержат адрес выполняемой инструкции.
\item Current Program Status Register (\R{CPSR}) — содержит флаги, о
писывающие текущее состояние процессора. Модифицируется при выполнении 
многих инструкций: логических, арифметических, и др.
\end{itemize}

Во всех режимах, кроме User mode и System mode, доступен также Saved 
Program Status Register (\Reg{SPSR}). После возникновения исключения регистр 
\Reg{CPSR}\ сохраняется в \Reg{SPSR}. Тем самым фиксируется состояние процессора (режим, 
состояние; флаги арифметических, логических операций, разрешения прерываний) 
на момент непосредственно перед прерыванием.


\part{\cmsis}
\section{Введение в \cmsis}\label{cmsisintro}

\cp{http://www.doulos.com/knowhow/arm/CMSIS/CMSIS\_Doulos\_Tutorial.pdf}



% \section{\cx}
% \chapter{Периферия}
% \section{Таймеры}
% \section{DMA}
% \section{UART}
% \section{CAN}
% \section{USB}
% \section{Ethernet}
% \chapter{Производители}
% \section{ST Microelectronics}
% \subsection{STM32F0xx /\cm{0}/}
% \subsection{STM32F1xx} \cm{1} VLDISCOVERY
\url{www.st.com/stm32-discovery}

% \subsection{STM32F4xx /\cm{4}/} STM32F4DISCOVERY
% 
% \include{milandr}
% 
% \section{NXP}
% \subsection{LPC210x}

\part{Программное обеспечение}

\chapter{Рабочая среда разработчика встраиваемых систем}


\begin{itemize}
  \item Операционная система с набором типовых утилит
  
  Для Windows требуется дополнительно установить несколько модулей из пакета
  \file{GnuWin32}, чтобы обеспечить минимальную совместимость с UNIX-средой.
  Установка \file{GnuWin32}\ описана в \labref{winsoftinstall}.
  
  Установка Linux описана в \labref{debianinstall}.
  
  \item САПР электронных устройств (EDA CAD)
  
  EDA используется для разработки схем, моделирования работы устройства,
  разводки печатных плат (ПП) и межплатных соединителей, и подготовки
  технологических файлов для производства ПП.
  
  \item САПР общего назначения
  
  В САПР создаются модели и чертежи конструкции устройств, прорабатывается
  компоновка, и проверяется работа электро-механических у    
  
  \item Система управления версиями (\term{VCS})
  
  VCS предназначены для хранения полной истории изменений файлов проекта, и
  позволяют получить выгрузку проекта на любой момент времени, вести несколько
  веток разработки, получить историю изменений конкретного файла, или сравнить
  две версии файла (\term{diff}).
  
  Установка VCS \git\ описана в \labref{gitinstall}.
  
  \item Текстовый редактор или интегрированная среда разработки (IDE)
  
  Редактирование текстов программ и скриптов сборки (компиляции) с
  цветовой подсветкой синтаксиса (в зависимости от языка файла),
  \term{автодополнением}\ и вызовом программ-утилит нажатием сочетаний 
  клавиш. Также включает различные вспомогательные функции, например
  отладочный интерфейс и отображение объектов программ.
  
  Установка IDE \eclipse\ описана в \labref{eclipseinstall}.
  
  \item Тулчайн
  
  Пакет кросс-компилятора, ассемблера, линкера и других утилит типа make,
  objdump,.. для получения прошивок из исходных текстов программ.
  
  Установка GNU toolchain описана в \labref{gnuinstall}.
  
  \item ПО для программатора, JTAG-адаптера
  
  Загрузка полученной прошивки в целевое устройство, редактирование памяти, 
  внутрисхемная отладка в процессе работы устройства, прямое измение сигналов на
  выводах процессора (граничное сканирование и тестирование железа).
  
  Установка ПО для адаптеров ST-Link \labref{stlinkinst}, Segger J-Link
  \labref{jlinkinst}.
  
  \item Симулятор для отладки программ без железа
  
  Симулятор может использоваться как ограниченная замена реального железа
  для начального обучения, и для отладки программ, не завязанных на работу
  железа.
  
  Установка QEMU \labref{qemuinstall}.
  
  \item Система верстки документации
  
  Для документирования проектов и написания руководств нужна система верстки
  документации, выполняющая трансляцию текстов программ и файлов 
  документации в выходной формат, чаще всего \file{.pdf} и \file{.html}.
  
  Установка \LaTeX\ \labref{texinstall}.
  
\end{itemize}



\labpart{Установка ПО}
\labwork{Установка Debian GNU/Linux}\label{debianinstall}


\labwork{Установка Git}\label{gitinstall}

Создадим рабочий каталог, установим систему контроля версий \git\ref{git}\ и 
получим локальную копию проекта этой книги, содержащий кроме текста для издательской системы
\LaTeX\ еще и исходные коды библиотек, примеры кода и т.п., которые вы захотите
использовать в своих проектах.

\bigskip\wcmd{\url{http://git-scm.com/download/win}}

Запуститься закачка установочного пакета scm-git (\file{Git-1.9.4-preview20140611.exe}), после его загрузки
запустите установщик, 

\bigskip
\menu{Welcome>Next}

\bigskip
\menu{GNU GPL>Next} 

\bigskip
\menu{Select components>Windows Explorer Integration>Simple Context Menu>Git GUI here>Next}

\bigskip
\menu{Use Git and optional Unix tools from the Command Prompt>Next}

\bigskip
\menu{Use OpenSSH>Next}

\bigskip
\menu{Checkout Windows-style>Next}

\bigskip
\menu{Extracting files...}

\bigskip
\menu{Completing Setup>\uncheckbox\ View ReleaseNotes>Finish}

\bigskip
Проверим что \git\ правильно установился:

\bigskip\wcmd{cmd}

\bigskip
\begin{lstlisting}[style=con]
C:\Documents and Settings\pda>git --version
git version 1.9.4.msysgit.0
\end{lstlisting}

\bigskip
Первое, что вам следует сделать после установки \git а\ ---указать ваше имя и
адрес электронной почты. Это важно, потому что каждый коммит в \git е содержит
эту информацию, и она включена в коммиты, передаваемые вами:
\begin{lstlisting}[style=con]
C:\Documents and Settings\pda>git config --global user.name "Vasya Pupkin"
C:\Documents and Settings\pda>git config --global user.email no@mail.com
C:\Documents and Settings\pda>git config --global push.default simple
\end{lstlisting}

\bigskip
Эти настройки достаточно сделать только один раз, поскольку в этом случае 
\git\ будет использовать эти данные для всего, что вы делаете.
 Если для каких-то отдельных проектов вы хотите указать другое имя или
электронную почту, можно выполнить эту же команду без параметра \verb|--global|
в каталоге с нужным проектом.

\bigskip
Создаем каталог \file{D:/ARM}\ и выгружаем текущую копию этой книги из
репозитория \url{https://github.com/ponyatov/CortexMx}, создавая
свой собственный локальный \term{репозиторий проекта}.

\bigskip\wcmd{cmd}

\bigskip
\begin{lstlisting}[style=con]
C:\Documents and Settings\pda>D:
D:\>mkdir \ARM
D:\>cd \ARM
D:\ARM>git clone --depth=1 https://github.com/ponyatov/CortexMx.git book
\end{lstlisting}

\labwork{Установка GNU toolchain}\label{gnuchaininstall}


\labwork{Установка утилит GnuWin32}\label{gnuwinstall}

\bigskip Для совместимости скриптов придется поставить несколько пакетов
из \file{GnuWin32}:

\bigskip\wcmd{\url{http://gnuwin32.sourceforge.net/packages.html}}

\bigskip\menu{\keys{Ctrl+F}>coreutils>\keys{Ctrl+Setup}}

\menu{\keys{Ctrl+F}>wget>\keys{Ctrl+Setup}}

\menu{\keys{Ctrl+F}>gnu make>\keys{Ctrl+Setup}}

\bigskip\file{coreutils-5.3.0.exe} основные UNIX-утилиты типа rm ls , собранные
под win32

\bigskip\menu{Welcome>Next}

\menu{License>Accept>Next}

\menu{Folder>\file{D:/ARM/GnuWin32}>Next}

\menu{Components>Next}

\menu{Start Menu>\file{GnuWin32/CoreUtils}>Next}

\menu{Select Additional>Next}

\menu{Ready to Install>Next}

\menu{Compliting>Finish}

\bigskip Аналогично ставим:

\file{make-3.81.exe} утилита make

\file{wget-1.11.4.exe} консольная утилита загрузки файлов по HTTP/FTP

\file{grep-2.5.4.exe} утилита поиска строк в файлах и stdin/stdout потоке

\labwork{Редактирование системной переменной Windows
\file{\$PATH}}\label{winpath}

\bigskip Чтобы утилиты \file{GnuWin32}\ были доступны, нужно прописать
переменную пользователя \verb|$PATH|\ в системном окружении.

\bigskip\menu{Пуск>Настройка>Панель управления>Система>Дополнительно>Переменные
среды}

\bigskip\menu{Переменные среды>переменные пользователя>Создать/Изменить}

\bigskip\menu{Имя переменной>PATH}

\menu{Значение переменной>добавить в начало
D:/ARM/GnuWin32/bin;D:/ARM/Yaga/bin;..}

\menu{Ok>Ok>Ok}

\bigskip Проверяем:
\begin{lstlisting}[style=con]
C:\Documents and Settings\pda>ls -la
total 3111
drwxr-xr-x   29 pda      user        0 Jul  4 14:03 .
drwxr-xr-x    9 pda      user        0 Oct  8  2013 ..
-rw-r--r--    1 pda      user     5242 May 22 14:29 .bash_history
drwxr-xr-x    2 pda      user        0 May 23  2013 .borland
drwxr-xr-x   18 pda      user        0 Sep  4  2013 .ccache
drwxr-xr-x    3 pda      user        0 Mar 26  2013 .eclipse
\end{lstlisting}
\begin{lstlisting}[style=con]
C:\Documents and Settings\pda>wget --version
GNU Wget 1.7

Copyright (C) 1995, 1996, 1997, 1998, 2000, 2001 Free Software Foundation, Inc.
This program is distributed in the hope that it will be useful,
but WITHOUT ANY WARRANTY; without even the implied warranty of
MERCHANTABILITY or FITNESS FOR A PARTICULAR PURPOSE.  See the
GNU General Public License for more details.

Originally written by Hrvoje Niksic <hniksic@arsdigita.com>.
\end{lstlisting}
\begin{lstlisting}[style=con]
C:\Documents and Settings\pda>make --version
GNU Make 3.81
Copyright (C) 2006  Free Software Foundation, Inc.
This is free software; see the source for copying conditions.
There is NO warranty; not even for MERCHANTABILITY or FITNESS FOR A
PARTICULAR PURPOSE.

This program built for i386-pc-mingw32
\end{lstlisting}



\labwork{Установка Java}\label{javainstall}

Для работы IDE \eclipse\ требуется установленная Java:

\bigskip\wcmd{\url{http://www.oracle.com/technetwork/java/javase/downloads/}}

\bigskip\begin{itemize}
  \item 
Минимальный вариант\ ---  ставим только Java Runtime:

\menu{Java Platform, Standard Edition>JRE>Download>Accept
License>\file{jre-8u5-windows-i586.exe}}

\menu{\file{jre-8u5-windows-i586.exe}>Welcome>\checkbox\ Change destination
folder>Install}

\menu{Destination folder>\file{D:/Java/jre8}>Next>Installing>Close}

  \item
Если вы планируете параллельно еще и осваивать язык Java\ --- ставим
Java SE JDK: 

\menu{Java Platform, Standard Edition>JDK>Download>Accept
License>\file{jdk-8u5-windows-i586.exe}}

\menu{\file{jdk-8u5-windows-i586.exe}>Welcome>Next}

\menu{Install to: \file{D:/Java/jdk8}>Next}

\menu{JRE Distination folder>Install to: \file{D:/Java/jre8}>Next}

\menu{Java SE Development Kit 8 Update 5 Successfully Installed>Close}

\end{itemize}


\labwork{Установка IDE \eclipse}\label{eclipseinstall}

\bigskip
Для работы IDE \eclipse\ требуется установленная Java \labref{javainstall}.

\bigskip
Для установки доступны два варианта:
\begin{enumerate}
\item \textbf{Eclipse Standard} базовый вариант среды, в ЛР рассмотрен именно он для иллюстрации 
ручной установки расширений
\item \textbf{Eclipse IDE for C/C++ Developers} вариант сборки 
уже включает расширение CDT, поэтому в следующий раз рекомендуем сразу качать его,
это упростит и съэкономит немного времени на установку рабочей среды
\end{enumerate}

\bigskip\wcmd{\url{http://www.eclipse.org/downloads/}}

\bigskip\menu{Eclipse Standard>Windows 32
Bit>Download>\file{eclipse-standard-luna-R-win32.zip}}

\bigskip Перетащите каталог \file{eclipse} из архива в \file{D:/ARM} и
создайте удобным для вас способом ссылку на \file{D:/ARM/eclipse/eclipse.exe}.

\bigskip\includegraphics[height=0.3\textheight]{fig/EclipseSplash.png}

\bigskip Workspace\ --- рабочий каталог, в котором создаются каталоги отдельных
проектов, типа \file{D:/WORK}. Eclipse создаст в нем служебный каталог
\file{.metadata}, и поместит в него служебную информацию, относящуюся сразу ко
всем проектам. Как побочный эффект, если в workspace уже есть какой-то каталог,
можно создать новый проект (например \file{book}), и в левой части рабочей
области \eclipse\ в окне \window{Project Explorer}\ появится дерево файлов
\file{book/*}.

\bigskip\menu{\file{D:/ARM/eclipse/eclipse.exe}>Workspace>\file{D:/ARM}>Use as
default>OK}

\bigskip\includegraphics[width=0.9\textwidth]{fig/EclipseMain.png}

\bigskip Проверяем наличие обновлений

\bigskip\menu{Help>Check for Updates>Details>No updates found>OK}

\bigskip В базовом варианте Eclipse поддерживает только Java, поэтому нужно
установить расширение для работы с С/C++: \file{CDT}.

\bigskip
Проект \file{CDT}\ предоставляет полнофункциональную интегрированную среду
для разработки на Си и \cpp. Поддерживаются: управление проектами и
компиляцией для различных тулчейнов, стандартная сборка через
\file{make}, навигация по исходным текстам, различные инструменты для
работы с иходным текстом, такие как иерархия типов, граф вызовов, браузер
подключаемых файлов, браузер макроопределений, редактор кода с подсветкой
синтаксиса, сворачивание синтаксических структур (фолдинг) и гипертекстовая
навигация, рефакторинг и генерация кода, средства визуальной отладки,
включающие просмотр памяти, регистров и дизассемблер.

\bigskip\wcmd{\url{http://www.eclipse.org/cdt/downloads.php}}

\bigskip Выделить и скопировать в буфер обмена ссылку

\file{p2 software repository}:
\url{http://download.eclipse.org/tools/cdt/releases/8.4}.

\bigskip Добавляем сетевое хранилище пакетов для \eclipse:

\bigskip\menu{\eclipse>Help>Install New Software>Work with>Add}

\bigskip\menu{Name>CDT}

\menu{Location>http://download.eclipse.org/tools/cdt/releases/8.4}

\menu{OK}

\bigskip
Выбрать (если оно не выбралось само) хранилище \menu{Work with:>CDT},
и в дереве выбора пакетов выбрать:

\bigskip
\dirtree{%
.1 CDT.
.2 CDT Main Features.
.3 \checkbox\ C/C++ Development Tools.
.2 CDT Optional Features.
.3 \checkbox\ C/C++ C99 LR Parser.
.3 \checkbox\ C/C++ GCC Cross Compiler Support.
.3 \checkbox\ C/C++ GDB Hardware Debugging.
}

\bigskip
\menu{Next>Next>Licenses>Accept>Finish}

\bigskip После установки пакетов появится окно с запросом перезапуска \eclipse.

\bigskip Аналогично ставим плагин GNU ARM Eclipse:

\bigskip
\menu{Help>Install>Work with>Add}

\menu{Name>GNU ARM plugin}

\menu{Location>\url{http://sourceforge.net/projects/gnuarmeclipse/files/Eclipse/updates/}}

\dirtree{%}
.1 GNU ARM C/C++ Cross Development Tools.
.2 \checkbox\ Cross Compiler Support.
.2 \checkbox\ Generic Cortex-M Project Template.
.2 \checkbox\ STM32Fx Project Templates.
.2 \checkbox\ OpenOCD Debugging Support.
}

\menu{Warning: You install unsigned content>Ok}

\bigskip
В \eclipse\ есть так называемые \term{перспективы} (perspective)\ --- это
переключаемые режимы отображения рабочего набора окон, настроенные под тип
работы. По умолчанию запускается перспектива \window{Java}. Нас
интересует перспектива \window{C/C++}:

\bigskip\menu{Window>Open Perspective>Other>C/C++>Ok}

\bigskip Также перспективу можно переключить кнопкой на панели в правом верхнем
углу:

\bigskip\includegraphics[width=0.9\textwidth]{fig/eclperpective.png}

\bigskip Для настройки привычных вам клавиш можно сразу зайти в
глобальные настроки среды и поменять привязку клавиш:

\bigskip\menu{Window>Preferences>General>Keys}

\menu{Type filter here:>F12}

\menu{Command}

\menu{Activate Editor>Binding>/удалить/}

\menu{Build Project>Binding>/нажать \keys{F12}/}

\menu{Apply>OK}



\labwork{Установка симулятора QEMU}\label{qemuinstall}

Нередко в практике разработчика возникают ситуации, когда программное обеспечение (ПО) для микроконтроллера
приходится писать в отсутствии под рукой аппаратной платформы.

Например, печатная плата устройства отдана на подготовку к производству, а времени ждать
готовое устройство для тестирования на нем программного обеспечения нет.

В таких случаях для оценки работоспособности ПО можно воспользоваться программным симулятором целевого микроконтроллера.

Для интегрированной среды разработки \eclipse\ CDT в качестве программного
симулятора микроконтроллеров ARM можно использовать симулятор (или виртуальную машину,если быть точным) 
\file{qemu-arm} с интерфейсом командной строки:

\bigskip\menu{\wcmd{\url{http://qemu.weilnetz.de/w32/}}>\file{qemu-w32-setup-20140702.exe}}

\menu{\file{qemu-w32-setup-20140702.exe}>Welcome>Next>License>Agree}

\menu{Choose Components}

\dirtree{%
.1 QEMU.
.2 \uncheckbox\ System Emulations.
.3 \checkbox\ arm.
.3 \checkbox\ armw.
}

\menu{Next}

\menu{Destination Folder>\file{D:/ARM/qemu}>Next>Finish}

\bigskip Добавьте \file{D:/ARM/qemu}\ в системную переменную
\file{\$PATH}\ (\labref{winpath}).

\bigskip

\begin{lstlisting}[style=con]
C:\Documents and Settings\pda>qemu-system-arm -version
C:\Documents and Settings\pda>cat D:\ARM\qemu\stdout.txt
QEMU emulator version 2.0.90, Copyright (c) 2003-2008 Fabrice Bellard
\end{lstlisting}


\labwork{Установка системы верстки документации \LaTeX}\label{texinstall}

Если вы планируете писать полноценную документацию на программы
и оборудование, или участовать в доделке этой книги, вы можете установить
систему верстки \LaTeX.

Для работы с \TeX\ требуется довольно приличное по усилиям
(само)обучение \cite{lugovsky}, но оно оправдывается если вы часто 
пишете документацию, особенно если в ней больше 10 формул.
Готовить документацию в MS Word\ --- (само)убийство мозга и времени,
идеология подстановочных макросов \TeX, богатый набор доп.пакетов
и командный ввод формул очень доставляют.

\win{Скачайте и установите пакет \miktex:}

\bigskip\wcmd{\url{http://miktex.org/download}>Other Downloads>Net Installer}

\menu{Save as:>\file{D:/ARM/soft/MikTeX/miktex-netsetup-2.9.4503}}

\bigskip Загрузка дистрибутивных файлов

\menu{\file{miktex-netsetup-2.9.4503}>License>Accept>Далее}

\menu{Task>Download>Далее}

Если у вас постоянное internet-соединение: \menu{Package Set>Basic MiKTeX>Далее}

Для offline работы\footnote{когда неизвестно какие пакеты понадобятся\ ---
\miktex\ умеет их докачивать по необходимости} \menu{Package
Set>Complete MikTeX>Далее}

\menu{Download Source>Russian Federation (ctan.uni-altai.ru)>Далее}

\menu{Distribution
Directory>\file{D:/ARM/soft/MikTeX}>Далее>Start>Executing>Далее>Close}

\bigskip Установка из ранее загруженного дистрибутива

\menu{\file{D:/ARM/soft/MikTeX/miktex-\alarm{netsetup}-2.9.4503}>License>Accept>Далее}

\menu{Task>Install>Далее>Basic MiKTeX>Далее}

\menu{Install for>Anyone/Only for user>Далее}

\menu{Install \alarm{from}:>\file{D:/ARM/soft/MikTeX}>Далее}

\menu{Install to:>\file{D:/LaTeX/MiKTeX}>Далее}

\menu{Settings}

\menu{Preferred paper>A4}

Важная опция: автоматическая докачка отсутствующих пакетов
\alarm{\menu{Install missing packages>Yes}}

\menu{Далее>Start>Executing>Close}

\bigskip Двухступенчатая установка позволяет сначала скачать полный дистрибутив
\miktex, а затем установить его на другой компьютер, не подключенный к
\internet, или c медленным/платным каналом не дающим взять и качнуть 200 Мб.

\bigskip Для удобной работы с \file{.tex} файлами в \eclipse\ нужно поставить 
дополнение \file{TeXlipse}:

\bigskip
\menu{\eclipse>Help>Install>Work with>Add}

\menu{Name>TeXlipse}

\menu{Location>\url{http://texlipse.sourceforge.net}}

\dirtree{%}
.1 TeXlipse.
.2 \checkbox\ TeXlipse.
}

%\include{ПО для JTAG-адаптера}

\chapter{Первые шаги}
\labpart{Первые шаги}
\labwork{Создание нового проекта в \eclipse}\label{labecreprj}

Создадим новый проект, напишем простую программу, и запустим ее в отладчике.

\bigskip\menu{\eclipse>File>New>Project>C/C++>C Project>Next}

\menu{Project name>hello}

\menu{Project type>Makefile project>Empty project}

\menu{Toolchains>Cross ARM GCC}

\menu{Finish}

\menu{This kind of project associated with C/C++ perspective>Yes}

 \bigskip В окне \window{Project Explorer}\ появится пустая закладка проекта
 \file{hello}. Если вдруг там будут какие-то файлы, значит кто-то до вас уже
 создал проект, и что-то туда наляпал. В этом случе повторите создание, задав
 имя типа \file{hello<номер группы><FIO>}\ или типа того, для полной
 уверенности можно сначала посмотреть что в \file{D:/ARM}\ нет папки с таким именем.

\labwork{Создание Makefile}\label{labmkmake}

Стоит объяснить, почему при создании проекта мы выбрали тип \file{Makefile
project}, хотя были доступны более логичные варианты типа \file{ARM C Project}.

Утилита \make\ ведет свою историю с 70х гг. Компьютеры тогда были большими,
тяжелыми, а главное медленными и с очень маленькой памятью (десятки$\div$сотни
Кб).
Компиляторам зачастую не хватало памяти, чтобы скомпилировать большую программу.
Кроме того, скорость их запуска и работы была тоже черепашьей.
Поэтому исходный код программы делили на модули, компилировали или
ассемблировали каждый модуль по-отдельности в \term{объектный код}, а затем уже
на конечном этапе с помощью \term{линкера}\ собирали несколько файлов объектного
кода в один исполнямый файл.

Для ускорения и упрощения этого процесса и была создана утилита \make.
Чтобы не вызывать лишний раз компилятор или какой-нибудь транслятор, в файле
\makefile\ прописываются зависимости между файлами. Затем запускается \make\ c
указанием какой файл нам нужно получить, и выполняется цепочка вызовов
нужных программ.

Следует отметить, что утилита \make\ используется до сих пор для сборки самых
современных программных пакетов (типа GCC 4.9.x), правда в комплексе с другими
средствами, обеспечивающими переносимость программ между разными ОС и
автогенерацией зависимостей из исходного кода.

\bigskip
Для наших целей \make\ используется как самое простое средство управления
компиляцией проекта. В средах разработки, особенно в коммерческих,
используются служебные файлы проектов, иногда бинарные, чаще текстовые, но
всегда запутанные и весьма развесистые.

Если вам вдруг понадобится откомпилировать ваш проект на другом компьютере,
с другой архитектурой, возможно вообще без графического
интерфейса\footnote{например какой-нибудь удаленный сервер на
процессоре 1995ВМ666 под раскряченным Solaris 7$\alpha$4, на котором лежит
криптобиблиотека, использующая при компиляции трофейный электро-механический 
энкодер, существующий в единственном экземпляре \smiley}, или вы вдруг решите
попробовать работать в другой IDE\ --- вы тут же вляпаетесь в ситуацию, 
когда нечем открыть файл проекта с заботливо прописанными опциями
компиляции.

\bigskip
\menu{\eclipse>\window{Project Explorer}>\file{hello}>\rms>Open Project}

\menu{\eclipse>\window{Project Explorer}>\file{hello}>\rms>New>File>File
name:>\file{Makefile}}

\bigskip
\lstinputlisting[style=mk,inputencoding=cp1251]{tmp/hello.mk}

\bigskip
Этот пример \makefile\ достаточно универсален и самодостаточен для большинства
проектов в этой книге. Кажущийся большой объем получился за счет использования
комментариев и переменных. И те, и другие служат для документирования проекта,
и повышают читаемость кода. В принципе никто не мешает\footnote{особенно для
микроскопических объемов исходных текстов программ для контроллеров\ ---
в самом худшем случае какие-то жалкие сотни Кб}\ написать несколько строк в
\file{.bat}нике с явным указанием опций компиляторам, или вообще откомпилировать
все исходники сразу одним вызовом \file{gcc}\ с кучей опций и списком исходных
файлов. Но если вам потребуется что-то изменить, куда проще и быстрее сделать
это в аккуратно оформленном самодокументированном \makefile.



\labwork{Hello World}\label{labhello}

Для начала нужно рассмотреть набор файлов минимального проекта:

\begin{itemize}

\item \file{README.txt}

Краткая информация о проекте\ --- название, авторы, обязательно ссылки на
\git-репозиторий, сайт, форум, и т.п.
 
\item \file{Makefile}

Файл с описанием зависимостей между файлами, настройками проекта (в переменных)
и правилами вызова компиляторов.

\item \file{startup.S}

Стартовый код процессора, включает инициализацию системы тактирования, мапинга
памяти, контроллера прерываний и минимальную инциализацию периферии.
Пишется на ассемблере, т.к. на Си получается слишком сложно, синтаксически
запутанно, или очень специфично для компилятора.

\item \file{init.c}

Сишный код инициализации железа (синтаксически легче описать блоки кода,
зависимые от целевого процессора).

\item \file{main.c}

Основной код, решающий поставленную задачу. 

\item \file{target.ld}

Скрипт линкера, настраивающий генерацию выходного бинарного файла в 
зависимости от целевого процессора\ --- прежде всего организация памяти,
и размещение сегментов кода/данных по фактическим адресам памяти.

\end{itemize}

\bigskip Создаем эти файлы аналогично \makefile\ в \labref{labmkmake}:

\bigskip\menu{\eclipse>\window{Project Explorer}>\file{hello}>\rms>New>File>File
name:>\file{НужныйФайл.xxx}}

\lstinputlisting[inputencoding=cp1251,title=README.txt]{hello/README.txt}


\labwork{Настройка отладчика в \eclipse}\label{labgdbinst}

\cp{http://makesystem.net/?p=2146}

 На сегодняшний день существуют много способов и инструментов для отладки
 embedded приложений, начиная с отладки “в железе” (внутрисхемная отладка)  и
 заканчивая всякими симуляторами. У каждого метода есть свои плюсы и минусы, но
 поскольку мы будим писать приложения для реальных устройств, то
 предпочтительней реальная отладка (в железе), то есть приложение будит
 исполняться непосредственно микроконтроллером.
 
Что нам понадобится для “железной отладки” :

\begin{itemize}
  \item 
ARM микроконтроллер (для симуляции необязателен)
  \item 
JTAG/SWD адаптер (для симуляции необязателен)
  \item 
GDB сервер (транслятор интерфейсов GDB/JTAG)
  \item 
GDB отладчик (имеет встроенный симулятор ARM7TDMI, используется для первых лаб) 
  \item 
плагин C/C++ GDB Hardware Debugging 
  \item 
плагин Eclipse Embedded Systems Register View 
\end{itemize}

\term{JTAG адаптер}\ (он же \term{программатор}) следует выбрать тот, который
поддерживает именно ваш микроконтроллер, а еще лучше, микроконтроллеры разных
производителей.
В моем случае (еще с давних времен у меня завалялись кристаллы от Texas
Instruments, ST Microelectronics, NXP, Atmel, Cypress), я сразу решил найти
программатор поддерживающий имеющиеся у меня камни. Порыскав в интернетах, мой
выбор пал на китайский клон знаменитого J-Link, в добавок к которому идет уйма
полезных утилит от Segger Microcontroller (тут обошлось без китая \smiley),
облегчающие жизнь разработчику.

В этой книге также рассмотрено несколько простых варинтов JTAG-адаптеров,
которые вы можете сделать сами, не обладая выдающимися знаниями в электронике
и технологиях производства печатных плат.

\bigskip
Структура аппаратно-программного комплекта для отладки:

\menu{микроконтроллер>JTAG/SWD>адаптер>LPT/USB>GDB сервер>протокол
GDB>GDB отладчик>IDE}

Адаптер подключается к выводам МК с помощью колодки (JTAG) или гребенки (SWD).

К компьютеру адаптер подключается через однин из распространенных интерфейсов:
совсем дешевые варинты ``на пяти резисторах'' через порт LPT, чуть подороже
через USB, совсем дорогие проф.модели могут иметь Ethernet интерфейс.

\bigskip
\term{Отладчик}\ (дебаггер, англ. debugger)\ --- компьютерная программа,
предназначенная для поиска ошибок в других программах. Отладчик позволяет
выполнять пошаговую трассировку, отслеживать, устанавливать или изменять
значения переменных в процессе выполнения кода, устанавливать и удалять
контрольные точки или условия остановки, сопоставлять двоичный код\ --- eгo
исходному тексту (на основе которых можно точно определить выполняемые
программой действия) и т.д. (Wiki)

Практически во все тулчейны входит утилита GDB (\file{arm-none-eabi-gdb}), это и
есть отладчик GNU. В принципе, дебаггер выполняет два типа действий:
управление исполнением программы в кристалле (через отладочный интерфейс) и
вывод результатов в консоль/графическую оболочку.

\bigskip
При сборке тулчайна (из исходников) невозможно заранее сказать, какой набор
отладочных средств будет у конечного пользователя\ --- у типичного
ембеддера\footnote{разработчика ПО под встраиваемые системы}\ запросто наберется
пара-тройка различных \jtag-адаптеровв, несколько демоплат со
встроенным адаптером, причем с разными процессорами, несколько собственных
устройств с самодельными отладочными интерфейсами, и еще для комплекта пару
чисто программных симуляторов \arm-ядер.

Задачу унификации интерфесов, и подключения всего этого зоопарка к одному и тому
же отладчику \gdb\ выполняет \term{\gdb-сервер}. Отладчик общается с сервером по
одному и тому же унифицированному \term{\gdb-протоколу}\ через последовательный
порт или TCP/IP соединение, а все сложности взаимодействия с железом берет на
себя сервер.
Это сделано (в том числе) для тех случаев, когда \gdb-отладчик работает на одном
ПК а \gdb-сервер на другом (в соседней комнате или соседнем государстве
\smiley). Опять же, сервер надо выбрать тот, который поддерживает ваш
\jtag-адапер или эмулятор.

\bigskip
\gdb, как и все остальные утилиты тулчайна, работает из командной строки, что
не всем удобно \smiley, поэтому для начала стоит научиться им пользоваться из
графической оболочки. \eclipse\ как и подобает серьёзной IDE, имеет средства
работы с \gdb, заменяющие его консоль, имеет графические кнопки для вызова
всех отладочных команд, и отображает содержимое регистров, памяти и т.п. в
графических окнах. Взаимодействие \eclipse/\gdb\ обеспечивает плагин \file{C/C++
GDB Hardware Debugging}, входящий в состав уже установленного ранее
расширения \file{CDT}.

\bigskip
Проверить наличие плагина можно так:

\menu{Help>Install New Software>Work with:>All Available Sites}

\menu{\uncheckbox\ Hide items that are already installes}

\menu{type filter>GDB}

\dirtree{%
.1 GDB.
.2 CDT Optional Features.
.3 \checkbox\ C/C++ GDB Hardware Debugging.
.2 Mobile and Device Development.
.3 \checkbox\ C/C++ GDB Hardware Debugging.
}

\bigskip

Прежде всего отключим оптимизацию кода проекта, задав в
\makefile\ значение переменной \file{OPTFLAGS = -O0}. 

Затем, нужно включить в проекте генерацию отладочной информации\footnote{имена
переменных, функций и т.п. объектов программы, в т.ч. и сами строки исходного
кода}, добавив опцию \file{-g[N]}.

Существуют три уровня отладочной информации:

\begin{enumerate}
\item в объектный код вставляется минимальный объем отладочной информации. Ее
вполне достаточно для трассировки вызовов функций и исследования глобальных
переменных, тем не менее, отсутствует информация для сопоставления выполняемого
кода со строками исходного кода и информация для отслеживания локальных
переменных.

\item используется по умолчанию. Помимо всей отладочной информации пepвoгo
уровня он дополнительно включает данные, необходимые для сопоставления строк
исходного кода с выполняемым кодом, а также имена и расположение локальных
переменных.

\item помимо всей отладочной информации пepвoгo и второго уровней, включает
дополнительную информацию, в частности определения макросов препроцессора.
\end{enumerate}

Используем 3 уровень, изменив в \makefile\ значение переменной 
\file{DEBFLAGS = -g3 -ggdb}. Опция \term{-ggdb}\ задает дополительно формат
отладочной информации. Доступны форматы STABS, DWARF2 и родной формат платформы.

В файле \file{startup.o.dump}\ при этом появляются дополнительные секции
с отладочной информацией, и в заголовок добавляются флаги, указывающие на
ее наличие:

\begin{lstlisting}[title=startup.o.dump]
startup.o:     file format elf32-littlearm
architecture: armv4t, flags 0x00000011: HAS_RELOC, HAS_SYMS
...
  4 .debug_line   00000044  00000000  00000000  000000b0  2**0
                  CONTENTS, RELOC, READONLY, DEBUGGING
  5 .debug_info   00000044  00000000  00000000  000000f4  2**0
                  CONTENTS, RELOC, READONLY, DEBUGGING
  6 .debug_abbrev 00000014  00000000  00000000  00000138  2**0
                  CONTENTS, READONLY, DEBUGGING
  7 .debug_aranges 00000020  00000000  00000000  00000150  2**3
                  CONTENTS, RELOC, READONLY, DEBUGGING
\end{lstlisting}   

\bigskip
Для настройки отладочного интерфейса заходим в меню

\menu{Run>Debug Configurations\ldots}

\menu{GDB Hardware Debugging>\rms>New}

В результате открывается окно с настройками отладки.

\bigskip
Для начала попробуем работу нашей прошивки на встроенном в \gdb\ программном
симуляторе процессора \file{ARM7TDMI}.

\bigskip
Вкладка Main.

\bigskip
В поле Name, можно дать имя всей конфигурации отладки, поскольку даже для одного
проекта бывают разные конфигурации отладки (скажем для отладки в RAM или Flash
памяти).

\menu{Name:>ARM7TDMI simulator}

В поле Project указываем имя проекта (поскольку в нашем workspace может быть
более одного проекта)

\menu{Project:>hello}

В поле C/C++ Application указываем имя *.elf файла сгенерированного после
компиляции (с введенными ранее настройками для Debug прошивки) проекта и который
будет использован во время отладки.

\menu{C/C++ Application:>startup.o}

\alarm{Перед тем как перейти к следующей вкладке, в нижней части окна
обязательно выбираем}

\menu{Legacy GDB Hardware Debugging Launcher}

\menu{Apply}

\bigskip
Вкладка Debugger. Здесь мы установим связь между отладчиком и графической
оболочкой, а также между отладчиком и \gdb-сервером (отсюда и название вкладки).

\bigskip
В поле GDB Command указываем имя отладчика из тулчайна.
Должен быть прописан в \file{\$PATH}, или можно указать полный путь

\menu{GDB Command:>arm-none-eabi-gdb}

В соответствии с идеологией ведущих разработчиков Free Software Foundation,
\gdb\ вместо собственного графического пользовательского интерфейса
предоставляет возможность подключения к внешним IDE, управляющим графическим
оболочкам либо использовать стандартный консольный текстовый интерфейс” (Wiki).
В общем, mi (Machine Interface) это протокол общения между отладчиком и
графической оболочкой.

\menu{Command Set:>Standard (Windows)}

\menu{Protocol Version:>mi}

Как ранее было сказано, общение между отладчиком и сервером осуществляется через
последовательный или TCP/IP порт, поэтому в общем случае следует выбирать опции
типа:

\menu{Remote Target>\checkbox\ Use remote target}

\menu{JTAG Device:>Generic TCP/IP}

\menu{IP address:>localhost}

\menu{Port number:>12345}

\alarm{Но поскольку мы собираемся использовать встроенный симулятор ARM7,
пока нужно \textbf{выключить} удаленную отладку:}

\menu{\uncheckbox\ Use remote target}

\bigskip

\menu{Apply}

\bigskip
Вкладка Startup (предписания отладчику перед началом работы).

\bigskip
Сброс необходим для того чтобы очистить регистры ARM процессора от значений
полученных в ходе предыдущей отладки (по желанию)

\menu{Reset and Delay (seconds):>3}

Останавливаем процессор для настройки эмулятора и загрузки отлаживаемой прошивки

\menu{\checkbox\ Halt}

В (пустом) текстовом поле вводим команды, выполняемые при старте отладки.

\bigskip
Настало время вернутся к вопросу об использовании симуляции микроконтроллеров
\file{ARM7TDMI}. На самом деле с этой задачей запросто справляется сам \gdb,
если указать ему стартовые команды:

\menu{target sim}

\menu{load}

%monitor speed auto
%monitor endian little
%monitor flash device = STM32F103RB
%monitor flash download = 1
%monitor flash breakpoints = 1

Из какого файла грузить прошивку

\menu{Load image>Use project binary}

Из какого файла грузить отладочную информацию\footnote{можно использовать
отдельный \file{.sym} файл}

\menu{Load symbols>Use project binary}

\menu{Apply}

\bigskip
Вкладка Common

\bigskip

\menu{Display in favorites menu>\checkbox\ Debug}

\menu{Standard Input and Output>\checkbox\ Allocate console}

\menu{\checkbox\ Launch in background}

\menu{Apply}

% \bigskip
% Задача \gdb-сервера слушать последовательный или TCP/IP порт и перенаправлять
% команды \jtag-адаптеру, а он в свою очередь микроконтроллеру.
% 
% Поскольку мы уже указали отладчику использовать локальный TCP/IP порт
% \file{localhost:12345}, осталось указать серверу какой отладочный интерфейс
% использовать.

\bigskip
При первом запуске отладки, \eclipse\ просит разрешение на переход в режим
отображения отладки (Debug perspective). Разрешаем и ставим галку “\checkbox\
больше не спрашивать“. Далее, открывается отображение многочисленных окон,
каждое со своим предназначением (окна исходного кода, окно дизассемблера, окно
отображения памяти и т.д.). При желании можно добавить различные окна через меню
\menu{Window>Show View}.

\bigskip

Первый запуск отладчика \bug: \menu{кнопка клопа>ARM7TDMI simulator}

Последующие запуски (последнего) отладчика: просто \keys{F11} 

\bigskip
\begin{lstlisting}
symbol-file C:\\ARM\\book\\hello\\startup.o
Reading symbols from C:\ARM\book\hello\startup.o...done.
target sim
load C:\\ARM\\book\\hello\\startup.o 
Connected to the simulator.
Loading section .text, size 0x50 vma 0x0
Start address 0x0
Transfer rate: 640 bits in <1 sec.
\end{lstlisting}


\chapter{Система управления версиями Git}

\label{git}\cite{progit}

Установку ПО см.\labref{gitinstall}

  Избегайте использования бинарных файлов, по возможности генерируйте их из
  текстового описания на каком-нибудь макроязыке\ --- в этом случае VCS
  обеспечит вам возможность получить историю или diff в
  человекочитаемопонимаемом виде, а не в виде набора невнятных кексов.
  
  Рекомендую использовать \git\ и один из проектных хостингов типа
  \url{https://github.com/}. Установка описана в \labref{gitinstall}.
  


\chapter{Интегрированная среда разработки \eclipse}
Например нажатием \keys{F3}\ в \eclipse\ можно переместится на
  определение функции, на имени которой находится текствый курсор.
  
    Автодополнение\ --- редактор предлагает варианты полного написания
  идентификаторов и ключевых слов по первым буквам и нажатию обычно
  \keys{Ctrl+Tab} или \keys{Ctrl+N}. Также автоматически расставляются
  закрывающие скобки, закрывающие операторы управляющих структур типа begin/end,
  и генерируются синтаксические элементы циклов при вводе ключевых слов
  if/for/while. Особенно удобно автодополнение при написании кода на ООП 
  языках\  --- при вводе имени класса или объекта и точки предлагается меню с
  именами данных и методов класса. 
  
  При вводе имени функции и скобки выводится всплывающее окно с подсказкой\ ---
  определение функции с типом возвращаемого значения, типом и именами
  параметров.
  
  Интерфейс IDE часто предусматривает различные вспомогательные окна,
  показывающие имена и свойства объектов, описанных в программе (переменные,
  функции, структуры,..), структуру проекта с зависимостями между файлами, блоки
  справки в зависимости от текущего выделенного элемента и т.п.
  
  Часто IDE имеет встроенный графический интерфейс для отладки программ,
  используя для этого интерфейсные библиотеки для программатора и
  специальный отладочный код, добавляемый к вашей программе при
  компиляции. Используя аппаратный модуль отладки на целевом процессоре и
  отладочный код, IDE обеспечивает отображение значений и изменений регистров
  процессора, состояние переферии, позволяет задать точки останова в программном
  коде, в т.ч. условные по значению или измениею переменных или регистров
  железа.
  При использовании ОС реального времени и системы аппаратной многозадачности
  отображается загрузка ядер, загрузка процессора и используемые ресурсы для
  каждой задачи, работа планировщика, и т.п.
  
Для удобной работы доступно несколько бесплатных вариантов IDE, далее
рассмотрим два варианта: тяжелая суперуниверсальная среда \eclipse, и легкая 
в отношении требуемых ресурсов системы CodeLite.
  


\chapter{Пакет кросс-компиляции GNU toolchain}

  Компилятор преобразует программы на языке программирования в \term{объектный
  код} (смесь кусочков машинного кода со служебной информацией) или в
  текст на языке ассемблера.
  
  \term{Кросс-компилятор}\ (arm-none-eabi-gcc) отличается от обычного
  компилятора тем, что генерирует код не для компьютера на котором он выполняется
  (\term{хост-система}, \verb|$HOST|), а для компьютера другой
  архитектуры\ --- \term{целевой} системы, \verb|$TARGET|.
  
  \term{Ассемблер}\ (as) преобразует человекочитаемый код программы в объектный
  код.
  
  \term{Линкер}\ (ld) объединяет несколько файлов объектного кода в один,
  и корректирует машинный код с учетом его конечного размещения в памяти
  целевой системы (адреса переменных, адреса переходов, размещение сегментов
  кода и данных в физической памяти целевой системы).
  
  \term{Дампер}\ (objdump) преобразует сегменты кода/данных из файла,
  полученного линкером, в формат, необходимый для ПО программатора: бинарные файлы, Intel
  HEX, ELF,.. загружаемые в масочное ПЗУ, FlashPROM (и EEPROM данных на МК
  ATmega).
  


\part{Отладка}

% \chapter{Рабочая среда разработчика встраиваемых систем}

\labpart{Установка ПО}

\section{Выбор и установка операционной системы}

\subsection{MacOS}

Этот странный пока случай не рассматриваем\ --- у меня нет под рукой Мака \smiley.

\subsection{Windows}

Самый распространенный вариант. Вам придется ограничиться этим вариантом если вам не повезло
с поставщиком контроллера: windows-only ПО поддержки, например софт для программатора, или
внезапно библиотеки только для коммерческих компиляторов. Аналогичная ситуация будет в случае
покупки какого-нибудь специфичного windows-only оборудования (лог.анализатор, измерительное 
оборудование или просто принтер).

Установку ОС не рассматриваем.

\win{Секции текста книги, зависяцие от ОС, будет выделены вот так.}

Для разработки встраиваемого ПО нужно поставить несколько пакетов, обеспечивающих
совместимость с UNIX средами \ref{winsoftinstall}.

\subsection{Linux}\label{linux}

\begin{enumerate}
\item \label{linuxref1}\linux\ удобен для разработчика, 
\item если он вам не удобен, см п.\ref{linuxref1}
\end{enumerate}

\lin{\labwork{Установка Debian GNU/Linux}\label{debianinstall}}

\lin{Секции текста книги, зависяцие от ОС, будет выделены вот так.}

\section{Установка инструментального программного обеспечения}

\win{\labwork{Установка инструментального ПО для Windows}\label{winsoftinstall}}

Создадим рабочий каталог, установим систему контроля версий \git\ref{git}\ и 
получим локальную копию проекта этой книги, содержащий кроме текста для издательской системы
\LaTeX\ еще и исходные коды библиотек, примеры кода и т.п., которые вы заходите использовать в своих
проектах:

\bigskip\wcmd{\url{http://git-scm.com/download/win}}

Запуститься закачка установочного пакета scm-git (\file{Git-1.9.4-preview20140611.exe}), после его загрузки
запустите установщик, 

\bigskip
\menu{Welcome>Next}, 

\bigskip
\menu{GNU GPL>Next}, 

\bigskip
\menu{Select components>Windows Explorer Integration>Simple Context Menu>Git GUI here>Next}

\bigskip
\menu{Use Git and optional Unix tools from the Command Prompt>Next}

\bigskip
\menu{Use OpenSSH>Next}

\bigskip
\menu{Checkout Windows-style>Next}

\bigskip
\menu{Extracting files...}

\bigskip
\menu{Completing Setup>[ ] View ReleaseNotes>Finish}

\bigskip
Проверим что \git\ правильно установился:

\bigskip\wcmd{cmd}

\bigskip
\begin{lstlisting}[style=con]
C:\Documents and Settings\pda>git --version
git version 1.9.4.msysgit.0
\end{lstlisting}

\bigskip
Создаем каталог \directory{D:/ARM}\ и выгружаем текущую копию этой книги из репозитория
\url{https://github.com/ponyatov/CortexMx}

\bigskip\wcmd{cmd}

\bigskip
\begin{lstlisting}[style=con]
C:\Documents and Settings\ponyatov>D:
D:\>mkdir \ARM
D:\>cd \ARM
D:\ARM>git clone --depth=1 https://github.com/ponyatov/CortexMx.git book
\end{lstlisting}

\begin{enumerate}
\item 
\end{enumerate}

\lin{\labwork{Установка инструментального ПО для \linux}\label{linsoftinstall}}

\chapter{Управление версиями при написании ПО}

\section{Git}\label{git}

% 
% \chapter{Eclipse / GNU toolchain}
% \section{Утилита Make}
% \section{binutils}
% \subsection{Ассемблер GNU AS}
% \subsection{Линкер LD}
% \subsection{Утилиты работа с файлами формата ELF}
% \section{Компилятор GCC}
% \section{IDE Eclipse}
% \chapter{Keil MDK-ARM}
% \chapter{IAR Embeded Workbench}
% 
% Великолепный вводный видеокурс по IAR и основам разработки на ARM
% 
% \menu{\cite{quantumleaps}>QuickStart>\href{http://youtu.be/3V9eqvkMzHA}{Lesson 0: Getting Started}}

\part{Встраиваемый \cpp}

\part{RTOS}
% \chapter{FreeRTOS}
% \chapter{Keil RTX}

\part{Автоматное программирование /фреймворк QuantumLeaps/}

\part{Разработка и изготовление железа}
\chapter{САПР KiCAD}
\chapter{Инструмент и оборудование}
\chapter{Технологии изготовления плат и монтажа}

\part{Подготовка документации}
\chapter{DocBook}
\chapter{\LaTeX}  Необходимо использовать человеко-читаемые простые текстовые файлы 
  (\file{plain ascii text}, кодировка по выбору, удобнее всего \file{utf8}) и
  использовать язык разметки\ --- DocBook, а удобнее всего \LaTeX.
  
  \alarm{
  Ни в коем случае не используйте для документации всякую бинарщину тип
  NarcoSoft Word\ --- текстовый формат необходим для корректной 
  и полноценной работы VCS. 
  Исключение по необходимости\ --- только графические файлы, подключаемые
  при генерации выходных файлов документации.
  }
  
  Эта книга написана с использованием языка разметки \LaTeX, и транслируется
  в экранный \file{.pdf} с помощью пакета \win{MiKTeX}/\lin{TeXlive}.
  
  Установка описана в \labref{texinstall}
  


\addcontentsline{toc}{part}{Литература}
\begin{thebibliography}{9}

\bibitem{leaps}{\copyright\ Quantum Leaps}

\bibitem{milandr}{\url{http://milandr.ru/} ЗАО <<ПКК Миландр>>}

\bibitem{progit}{\url{http://git-scm.com/book/ru} перевод:
Scott Chacon
\textbf{Pro Git}
}

\bibitem{habraQP}{\url{http://habrahabr.ru/post/114239/} хабра: Quantum Leaps QP
и диаграммы состояний в UML}

\bibitem{quantumleaps}{\url{http://www.state-machine.com/}
Quantum$^{\circledR}L^{e}aPs$ State Machines \& Tools}

\end{thebibliography}


\end{document}
