\documentclass[oneside,12pt]{book}

\usepackage[utf8]{inputenc}
\usepackage[english,russian]{babel}
\usepackage{indentfirst}

%\usepackage[paperwidth=15cm,paperheight=7.5cm]{geometry} % планшет
\usepackage[paperwidth=297mm,paperheight=210mm]{geometry} % A4
%\usepackage[paperwidth=148mm,paperheight=105mm]{geometry} % A5
\geometry{left=1cm,right=1cm,top=1cm,bottom=1cm}

%\usepackage[colorlinks=true,
%]{hyperref}
%\newcommand{\email}[2]{#1\ \href{mailto:#2}{<\nolinkurl{#2}>}}
% \newcommand{\url}[1]{\emph{#1}}
% %\newcommand{\email}[2]{\emph{#1\ <#2>}}
\usepackage[unicode,colorlinks=true,
pdftitle={Azbuka ARMaturschika (ru)},
pdfauthor={(c) Dmitry Ponyatov <dponyatov@gmail.com>, SSAU ASCL},
pdfsubject={ru manual on writing programs for Cortex-M MCUs},
pdfkeywords={ARM} {Cortex} {MCU} {ARMatura} {Arduino} {SSAU}
]{hyperref}

\newcommand{\cm}[1]{Cortex-M#1}
\newcommand{\cx}{\cm{x}}

\newcommand{\vld}{STM32VLDISCOVERY}

\renewcommand{\url}[1]{\textbf{#1}}
\newcommand{\email}[1]{$<$\textbf{#1}$>$}

\newcommand{\cpp}{$C^{+^{+}}$}

\newcommand{\cp}[1]{\footnote{копипаста: #1}}

\begin{document}

\title{\Huge{Азбука ARMатурщика}\\
\normalsize{учебный курс по микроконтроллерам \cx:\\
Миландр 1986ВЕ, STM32F, LPC21xx}}
\author{\copyright\\
Понятов Д.А. \email{dponyatov@gmail.com}, ИКП СГАУ, \\
Недяк С.П. \email{fvs@fet.tusur.ru}, ТУСУР
}
\maketitle
\tableofcontents

\part{Обзор семейства микроконтроллеров \cx}

\cp{\url{http://ru.wikipedia.org/wiki/ARM\_(архитектура)}}

Архитектура ARM (Advanced RISC Machine, Acorn RISC Machine, 
усовершенствованная RISC-машина)\ --- семейство лицензируемых 32-битных и 
64-битных микропроцессорных ядер разработки компании ARM Limited.

Среди лицензиатов: практически все заметные разработчики цифровых
электронных компонентов.
Многие лицензиаты делают собственные версии ядер на базе ARM.

Значимые семейства процессоров: ARM7, ARM9, ARM11 и Cortex.

В 2007 году около 98\% из более чем миллиарда мобильных телефонов, продаваемых 
ежегодно, были оснащены по крайней мере одним процессором ARM. По состоянию 
на 2009 на процессоры ARM приходилось до 90\% всех встроенных 32-разрядных 
процессоров. Процессоры ARM широко используются в потребительской 
электронике\ --- в том числе КПК, мобильных телефонах, цифровых носителях и 
плеерах, портативных игровых консолях, калькуляторах и компьютерных п
ериферийных устройствах, таких как жесткие диски или маршрутизаторы.

Эти процессоры имеют низкое энергопотребление, поэтому находят широкое 
применение во встраиваемых системах и преобладают на рынке мобильных 
устройств, для которых данный фактор немаловажен.

В настоящее время значимыми являются несколько семейств процессоров ARM:

\begin{itemize}
\item ARM7 (с тактовой частотой до 60-72 МГц), предназначенные, например, для 
недорогих мобильных телефонов и встраиваемых решений средней производительности. 
В настоящее время активно вытесняется новым семейством Cortex.
\item ARM9, ARM11 (с частотами до 1 ГГц) для продвинутых телефонов, карманных 
компьютеров и встраиваемых решений высокой производительности.
\item Cortex A\ --- новое семейство процессоров на смену ARM9 и ARM11.
\item Cortex M\ --- новое семейство процессоров на смену ARM7, также 
призванное занять новую для ARM нишу встраиваемых решений низкой 
производительности. В семействе присутствуют три значимых ядра: \cm{0}, 
\cm{3} и \cm{4}.
\end{itemize}

\chapter{Архитектура ядра}
\chapter{Периферия}
\section{Таймеры}
\section{DMA}
\section{UART}
\section{CAN}
\section{USB}
\section{Ethernet}
\chapter{Производители}
\section{ST Microelectronics}
\subsection{STM32F0xx} \cm{0}
\subsection{STM32F1xx} \cm{1} VLDISCOVERY
\url{www.st.com/stm32-discovery}

\subsection{STM32F4xx} \cm{4} STM32F4DISCOVERY
\section{ЗАО <<ПКК Миландр>>}

Компания <<Миландр>> выпускает специализированные МК для спецприменений,
и (относительно) дешевый вариант МК в пластиковом корпусе для обучения
и неответственных применений.

\bigskip

Телефон: (495) 981-54-33 (8.30-17.00, отдел маркетинга и продаж *)

Факс: (495) 981-54-36

E-mail: \email{info@milandr.ru}

Сайт: \url{http://www.milandr.ru}

Форум: \url{http://forum.milandr.ru}

Адрес: 124498, г. Москва, Зеленоград, проезд 4806, дом 6

\subsection{КР1986ВЕ91Т}
\subsection{MDR1986Q1}
\section{NXP}
\subsection{LPC210x}

\part{Программное обеспечение}
\chapter{Eclipse / GNU toolchain}
\section{Утилита Make}
\section{binutils}
\subsection{Ассемблер GNU AS}
\subsection{Линкер LD}
\subsection{Утилиты работа с файлами формата ELF}
\section{Компилятор GCC}
\section{IDE Eclipse}
\chapter{Keil MDK-ARM}
\chapter{IAR Embeded Workbench}

\part{Встраиваемый \cpp}

\part{Лабораторные работы}

\part{Литература}

\end{document}
