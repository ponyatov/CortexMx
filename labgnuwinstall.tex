\labwork{Установка утилит GnuWin32}\label{gnuwinstall}

\bigskip Для совместимости скриптов придется поставить несколько пакетов
из \file{GnuWin32}:

\bigskip\wcmd{\url{http://gnuwin32.sourceforge.net/packages.html}}

\bigskip\menu{\keys{Ctrl+F}>coreutils>\keys{Ctrl+Setup}}

\menu{\keys{Ctrl+F}>wget>\keys{Ctrl+Setup}}

\menu{\keys{Ctrl+F}>gnu make>\keys{Ctrl+Setup}}

\bigskip\file{coreutils-5.3.0.exe} основные UNIX-утилиты типа rm ls , собранные
под win32

\bigskip\menu{Welcome>Next}

\menu{License>Accept>Next}

\menu{Folder>\file{D:/ARM/GnuWin32}>Next}

\menu{Components>Next}

\menu{Start Menu>\file{GnuWin32/CoreUtils}>Next}

\menu{Select Additional>Next}

\menu{Ready to Install>Next}

\menu{Compliting>Finish}

\bigskip Аналогично ставим:

\file{make-3.81.exe} утилита make

\file{wget-1.11.4.exe} консольная утилита загрузки файлов по HTTP/FTP

\file{grep-2.5.4.exe} утилита поиска строк в файлах и stdin/stdout потоке

\labwork{Установка MinGW}\label{labinstmingw}

\bigskip
\alarm{Под Windows x64 обнаружилась проблема\ --- если в
\makefile\ используется перенаправление вывода команды в файл типа}
\verb|objdump -xd program.o > program.o.dump| \alarm{ или маркеры сцепления, при
выполнении
\file{make} (из пакета \file{GnuWin32}) завершается с ошибкой}

\verb|make: Interrupt/Exception caught| \verb|(code = 0xc00000fd,|
\verb|addr = 0x4227d3)|\alarm{. 

Для обхода этой проблемы был найден 
способ\ --- поставить пакет \file{MinGW}: это GNU тулчейн для нативной
компиляции программ под win32. За счет небольшой потери объема диска получаем
решение проблемы с \file{make}\ на Windows x64, и заодно получаем возможность
компилировать простые вспомогательные утилиты на Си/\cpp.}

Кроме того, нужно стараться писать \term{портабельный}\ код максимально
независимый от платформы\footnote{чтобы можно было по необходимость легко и
быстро перенести ваш проект на любой другой микроконтроллер, или использовать
одни и те же куски кода как на МК, так и на ПК, например процедуры
кодирования/декодирования данных или реализаци протоколов обмена данными}, а
тестировать платформо-независимость можно, компилируя один и тот же код
одновременно и для микроконтроллера, и для выполнения под Windows.

\bigskip
\alarm{Для совместимости поставим 32-битную версию \file{MinGW}.}

\bigskip
\wcmd{\url{http://www.mingw.org/}}

\menu{\keys{Download Installer}>\file{mingw-get-setup.exe}>Install}

\menu{Installation Directory>\file{D:/MinGW}>Continue}

\labwork{Редактирование системной переменной Windows
\file{\$PATH}}\label{winpath}

\bigskip Чтобы утилиты \file{GnuWin32}\ были доступны, нужно прописать
переменную пользователя \verb|$PATH|\ в системном окружении.

\bigskip\menu{Пуск>Настройка>Панель управления>Система>Дополнительно>Переменные
среды}

\bigskip\menu{Переменные среды>переменные пользователя>Создать/Изменить}

\bigskip\menu{Имя переменной>PATH}

\menu{Значение переменной>добавить в начало
D:/ARM/GnuWin32/bin;D:/MinGW/bin;D:/ARM/Yaga/bin;..}

\menu{Ok>Ok>Ok}

\bigskip Проверяем:
\begin{lstlisting}[style=con]
C:\Documents and Settings\pda>ls -la
total 3111
drwxr-xr-x   29 pda      user        0 Jul  4 14:03 .
drwxr-xr-x    9 pda      user        0 Oct  8  2013 ..
-rw-r--r--    1 pda      user     5242 May 22 14:29 .bash_history
drwxr-xr-x    2 pda      user        0 May 23  2013 .borland
drwxr-xr-x   18 pda      user        0 Sep  4  2013 .ccache
drwxr-xr-x    3 pda      user        0 Mar 26  2013 .eclipse
\end{lstlisting}
\begin{lstlisting}[style=con]
C:\Documents and Settings\pda>wget --version
GNU Wget 1.7

Copyright (C) 1995, 1996, 1997, 1998, 2000, 2001 Free Software Foundation, Inc.
This program is distributed in the hope that it will be useful,
but WITHOUT ANY WARRANTY; without even the implied warranty of
MERCHANTABILITY or FITNESS FOR A PARTICULAR PURPOSE.  See the
GNU General Public License for more details.

Originally written by Hrvoje Niksic <hniksic@arsdigita.com>.
\end{lstlisting}
\begin{lstlisting}[style=con]
C:\Documents and Settings\pda>mingw32-make --version
GNU Make 3.82.90
Built for i686-pc-mingw32
Copyright (C) 1988-2012 Free Software Foundation, Inc.
License GPLv3+: GNU GPL version 3 or later <http://gnu.org/licenses/gpl.html>
This is free software: you are free to change and redistribute it.
There is NO WARRANTY, to the extent permitted by law.
\end{lstlisting}
