\documentclass[oneside,14pt]{book}

\usepackage[T1,T2A]{fontenc}
\usepackage[utf8]{inputenc}
\usepackage[english,russian]{babel}
\usepackage{indentfirst}

%\usepackage[paperwidth=15cm,paperheight=7.5cm]{geometry} % планшет
%\usepackage[paperwidth=297mm,paperheight=210mm]{geometry} % A4
%\usepackage[paperwidth=148mm,paperheight=105mm]{geometry} % A5
\usepackage[paperwidth=18cm,paperheight=13cm,margin=5mm]{geometry} % экран*2

%\usepackage[colorlinks=true,
%]{hyperref}
%\newcommand{\email}[2]{#1\ \href{mailto:#2}{<\nolinkurl{#2}>}}
% %\newcommand{\email}[2]{\emph{#1\ <#2>}}
\usepackage[unicode,colorlinks,
pdftitle={Azbuka ARMaturschika (ru)},
pdfauthor={(c) Dmitry Ponyatov <dponyatov@gmail.com>, SSAU ASCL},
pdfsubject={ru manual on writing programs for Cortex-M MCUs},
pdfkeywords={ARM} {Cortex} {MCU} {ARMatura} {Arduino} {SSAU}
]{hyperref}

\usepackage{wrapfig}
\usepackage{graphicx}
\usepackage{epstopdf}
\DeclareGraphicsExtensions{.eps}

\usepackage{listings}
\usepackage{dirtree}
\usepackage[usenames,dvipsnames,svgnames]{xcolor}
\newcommand{\cppcolor}{\color[rgb]{0.94, 0.97, 1.0}} % Alice blue
\newcommand{\asmcolor}{\color[rgb]{0.98, 0.92, 0.84}} % Antique white
\newcommand{\concolor}{\color[rgb]{0.88, 1.0, 1.0}} % Light cyan

%\definecolor{cppcolor}{rgb}{0.94, 0.97, 1.0}
\lstset{frame=single,
numbers=left, numberstyle=\small,
commentstyle=\color{cyan}\texttt,
tabsize=4,
keywordstyle=\color{blue}\texttt
}
\usepackage{lstlangarm}

\lstdefinestyle{cpp}{language=C++,backgroundcolor=\cppcolor}
\lstdefinestyle{asm}{language={[ARM]Assembler},backgroundcolor=\asmcolor}
\lstdefinestyle{con}{backgroundcolor=\concolor}

\newcommand{\cm}[1]{Cortex-M#1}
\newcommand{\cx}{\cm{x}}

\newcommand{\vld}{STM32VLDISCOVERY}

%\renewcommand{\url}[1]{\textbf{#1}}
\newcommand{\email}[1]{$<$\href{mailto:#1}{\textbf{#1}}$>$}

\newcommand{\cpp}{$C^{+^{+}}$}

\newcommand{\cp}[1]{\footnote{копипаста: #1}}

\newcommand{\thmod}{Thumb}
\newcommand{\armod}{ARM}

\newcommand{\Reg}[1]{\textbf{#1}}
\newcommand{\R}[1]{\Reg{R#1}}

\newcommand{\periph}[1]{\texttt{#1}}
\newcommand{\jtag}{\periph{JTAG}}

\usepackage{wasysym} % smileys
\usepackage{gensymb} % celsius
\usepackage{amssymb} % windows key
\usepackage{textcomp} % bigcircle

\usepackage[os=win]{menukeys}
\newcommand{\winstart}{$\boxplus$}
\newcommand{\file}[1]{\textbf{\textsf{#1}}}
\newcommand{\window}[1]{\textbf{\textit{#1}}}
\newcommand{\alarm}[1]{{\color{DarkRed}#1}}
\newcommand{\wcmd}[1]{\keys{\winstart+R}\ \directory{#1}}
\newcommand{\checkbox}{$\boxtimes$}
\newcommand{\uncheckbox}{$\square$}

\newcommand{\win}[1]{\includegraphics[height=10ex]{fig/winlogo.jpg} #1}
\newcommand{\lin}[1]{\includegraphics[height=10ex]{fig/linuxcolor.png} #1}

\newcommand{\linux}{Linux}
\newcommand{\git}{Git}
\newcommand{\eclipse}{\textcircled{$\equiv$}\textsc{eclipse}}
%\newcommand{\term}[1]{\underline{#1}}
\newcommand{\term}[1]{\underline{\color{DarkBlue} #1}}

\usepackage{tocloft}
\newcommand{\listlabname}{Лабораторные работы}
\newlistof{lab}{ex}{\listlabname}
\newcommand{\labpart}[1]{\addcontentsline{ex}{part}{#1}}
\newcounter{labworkcounter}
\newcommand{\labwork}[1]{
\refstepcounter{labworkcounter}
\section*{ЛР\thelabworkcounter: #1}
\addcontentsline{toc}{subsection}{ЛР\thelabworkcounter: #1}
\addcontentsline{ex}{section}{ЛР\thelabworkcounter: #1}
}
\newcommand{\labref}[1]{ЛР\ref{#1}}

\newcommand{\mytitle}[1]{
\title{\Huge{Азбука халтурщика-ARMатурщика}\\
#1\\
\normalsize{учебный курс по микроконтроллерам \cx:\\
Миландр 1986ВЕ, STM32F, LPC21xx}}
}

\author{\copyright\\
Понятов Д.А. \email{dponyatov@gmail.com}, ИКП СГАУ, \\
Недяк С.П. \email{fvs@fet.tusur.ru}, ТУСУР
}


\begin{document}

Общая структура рабочей среды разработчика встроенных систем:

\begin{itemize}
  \item Операционная система с набором типовых утилит
  
  Для Windows требуется дополнительно установить несколько модулей из пакета
  \file{GnuWin32}, чтобы обеспечить минимальную совместимость с UNIX-средой.
  
  \item Пакет кросс-компилятора и утилит типа make, objdump,..
  
  Компилятор преобразует программы на языке программирования в \term{объектный
  код} (смесь кусочков машинного кода со служебной информацией) или в
  текст на языке ассемблера.
  
  \term{Кросс-компилятор}\ отличается от обычного компилятора тем, что
  генерирует код не для компьютера на котором он выполняется
  (\term{хост-система}, \verb|$HOST|), а для компьютера другой
  архитектуры\ --- \term{целевой} системы, \verb|$TARGET|.
  
  \item ПО для программатора, JTAG-адаптера
  \item Текстовый редактор или интегрированная среда разработки (IDE)
  \item ПО адаптера JTAG
\end{itemize}

\win{\labwork{Установка IDE}\label{winide}}

Для удобной работы доступно несколько бесплатных вариантов IDE (интегрированных сред разработки),
рассмотрим два варианта: тяжелая суперуниверсальная среда Eclipse, и легкая в отношении требуемых
ресурсов системы CodeLite.

\bigskip Для работы Eclipse требуется установленная Java:

\bigskip\wcmd{\url{http://www.oracle.com/technetwork/java/javase/downloads/}}

\bigskip\begin{itemize}
  \item 
Минимальный вариант\ ---  ставим только Java Runtime:

\menu{Java Platform, Standard Edition>JRE>Download>Accept
License>\file{jre-8u5-windows-i586.exe}}

\menu{\file{jre-8u5-windows-i586.exe}>Welcome>\checkbox\ Change destination
folder>Install}

\menu{Destination folder>\file{D:/Java/jre8}>Next>Installing>Close}

  \item
Если вы планируете параллельно еще и осваивать язык Java\ --- ставим
Java SE JDK: 

\menu{Java Platform, Standard Edition>JDK>Download>Accept
License>\file{jdk-8u5-windows-i586.exe}}

\menu{\file{jdk-8u5-windows-i586.exe}>Welcome>Next}

\menu{Install to: \file{D:/Java/jdk8}>Next}

\menu{JRE Distination folder>Install to: \file{D:/Java/jre8}>Next}

\menu{Java SE Development Kit 8 Update 5 Successfully Installed>Close}

\end{itemize}

Для установки доступны два варианта:
\begin{enumerate}
\item \textbf{Eclipse Standard} базовый вариант среды, в ЛР рассмотрен именно он для иллюстрации 
ручной установки расширений
\item \textbf{Eclipse IDE for C/C++ Developers} вариант сборки 
уже включает расширение CDT, поэтому в следующий раз рекомендуем сразу качать его,
это упростит и съэкономит немного времени на установку рабочей среды
\end{enumerate}

\bigskip\wcmd{\url{http://www.eclipse.org/downloads/}}

\bigskip\menu{Eclipse Standard>Windows 32
Bit>Download>\file{eclipse-standard-luna-R-win32.zip}}

\bigskip Перетащите каталог \file{eclipse} из архива в \file{D:/ARM} и
создайте удобным для вас способом ссылку на \file{D:/ARM/eclipse/eclipse.exe}.

\bigskip\includegraphics[height=0.3\textheight]{fig/EclipseSplash.png}

\bigskip Workspace\ --- рабочий каталог, в котором создаются каталоги отдельных
проектов, типа \file{D:/WORK}. Eclipse создаст в нем служебный каталог
\file{.metadata}, и поместит в него служебную информацию, относящуюся сразу ко
всем проектам. Как побочный эффект, если в workspace уже есть какой-то каталог,
можно создать новый проект (например \file{book}), и в левой части рабочей
области \eclipse\ в окне \window{Project Explorer}\ появится дерево файлов
\file{book/*}.

\bigskip\menu{\file{D:/ARM/eclipse/eclipse.exe}>Workspace>\file{D:/ARM}>Use as
default>OK}

\bigskip\includegraphics[width=0.9\textwidth]{fig/EclipseMain.png}

\bigskip Проверяем наличие обновлений

\bigskip\menu{Help>Check for Updates>Details>No updates found>OK}

\bigskip В базовом варианте Eclipse поддерживает

\end{document}
