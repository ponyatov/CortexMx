\labwork{Установка симулятора QEMU}\label{qemuinstall}

Нередко в практике разработчика возникают ситуации, когда программное обеспечение (ПО) для микроконтроллера
приходится писать в отсутствии под рукой аппаратной платформы.

Например, печатная плата устройства отдана на подготовку к производству, а времени ждать
готовое устройство для тестирования на нем программного обеспечения нет.

В таких случаях для оценки работоспособности ПО можно воспользоваться программным симулятором целевого микроконтроллера.

Для интегрированной среды разработки \eclipse\ CDT в качестве программного
симулятора микроконтроллеров ARM можно использовать симулятор (или виртуальную машину,если быть точным) 
\file{qemu-arm} с интерфейсом командной строки:

\bigskip\menu{\wcmd{\url{http://qemu.weilnetz.de/w32/}}>\file{qemu-w32-setup-20140702.exe}}

\menu{\file{qemu-w32-setup-20140702.exe}>Welcome>Next>License>Agree}

\menu{Choose Components}

\dirtree{%
.1 QEMU.
.2 \uncheckbox\ System Emulations.
.3 \checkbox\ arm.
.3 \checkbox\ armw.
}

\menu{Next}

\menu{Destination Folder>\file{D:/ARM/qemu}>Next>Finish}

\bigskip Добавьте \file{D:/ARM/qemu}\ в системную переменную
\file{\$PATH}\ (\labref{winpath}).

\bigskip

\begin{lstlisting}[style=con]
C:\Documents and Settings\pda>qemu-system-arm -version
C:\Documents and Settings\pda>cat D:\ARM\qemu\stdout.txt
QEMU emulator version 2.0.90, Copyright (c) 2003-2008 Fabrice Bellard
\end{lstlisting}

