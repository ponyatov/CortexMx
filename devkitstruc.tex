
\begin{itemize}
  \item Операционная система с набором типовых утилит
  
  Для Windows требуется дополнительно установить несколько модулей из пакета
  \file{GnuWin32}, чтобы обеспечить минимальную совместимость с UNIX-средой.
  Установка \file{GnuWin32}\ описана в \labref{winsoftinstall}.
  
  Установка Linux описана в \labref{debianinstall}.
  
  \item САПР электронных устройств (EDA CAD)
  
  EDA используется для разработки схем, моделирования работы устройства,
  разводки печатных плат (ПП) и межплатных соединителей, и подготовки
  технологических файлов для производства ПП.
  
  \item САПР общего назначения
  
  В САПР создаются модели и чертежи конструкции устройств, прорабатывается
  компоновка, и проверяется работа электро-механических у    
  
  \item Система управления версиями (\term{VCS})
  
  VCS предназначены для хранения полной истории изменений файлов проекта, и
  позволяют получить выгрузку проекта на любой момент времени, вести несколько
  веток разработки, получить историю изменений конкретного файла, или сравнить
  две версии файла (\term{diff}).
  
  Установка VCS \git\ описана в \labref{gitinstall}.
  
  \item Текстовый редактор или интегрированная среда разработки (IDE)
  
  Редактирование текстов программ и скриптов сборки (компиляции) с
  цветовой подсветкой синтаксиса (в зависимости от языка файла),
  \term{автодополнением}\ и вызовом программ-утилит нажатием сочетаний 
  клавиш. Также включает различные вспомогательные функции, например
  отладочный интерфейс и отображение объектов программ.
  
  Установка IDE \eclipse\ описана в \labref{eclipseinstall}.
  
  \item Тулчайн
  
  Пакет кросс-компилятора, ассемблера, линкера и других утилит типа make,
  objdump,.. для получения прошивок из исходных текстов программ.
  
  Установка GNU toolchain описана в \labref{gnuinstall}.
  
  \item ПО для программатора, JTAG-адаптера
  
  Загрузка полученной прошивки в целевое устройство, редактирование памяти, 
  внутрисхемная отладка в процессе работы устройства, прямое измение сигналов на
  выводах процессора (граничное сканирование и тестирование железа).
  
  Установка ПО для адаптеров ST-Link \labref{stlinkinst}, Segger J-Link
  \labref{jlinkinst}.
  
  \item Симулятор для отладки программ без железа
  
  Симулятор может использоваться как ограниченная замена реального железа
  для начального обучения, и для отладки программ, не завязанных на работу
  железа.
  
  Установка QEMU \labref{qemuinstall}.
  
  \item Система верстки документации
  
  Для документирования проектов и написания руководств нужна система верстки
  документации, выполняющая трансляцию текстов программ и файлов 
  документации в выходной формат, чаще всего \file{.pdf} и \file{.html}.
  
  Установка \LaTeX\ \labref{texinstall}.
  
\end{itemize}

