\section{Общая структура рабочей среды разработчика встроенных систем}

\begin{itemize}
  \item Операционная система с набором типовых утилит
  
  Для Windows требуется дополнительно установить несколько модулей из пакета
  \file{GnuWin32}, чтобы обеспечить минимальную совместимость с UNIX-средой.
  Установка \file{GnuWin32}\ описана в \labref{winsoftinstall}.
  
  Установка Linux описана в \labref{debianinstall}.
  
  \item Система управления версиями (\term{VCS})
  
  Рекомендую использовать \git\ и один из проектных хостингов типа
  \url{https://github.com/}. Установка описана в \labref{gitinstall}.
  
  VCS предназначены для хранения полной истории изменений файлов проекта, и
  позволяют получить выгрузку проекта на любой момент времени, вести несколько
  веток разработки, получить историю изменений конкретного файла, или сравнить
  две версии файла (\term{diff}).
  
  Избегайте использования бинарных файлов, по возможности генерируйте их из
  текстового описания на каком-нибудь макроязыке\ --- в этом случае VCS
  обеспечит вам возможность получить историю или diff в
  человекочитаемопонимаемом виде, а не в виде набора невнятных кексов.
  
  \item Текстовый редактор или интегрированная среда разработки (IDE)
  
  Редактирование текстов программ и скриптов сборки (компиляции) с
  цветовой подсветкой синтаксиса (в зависимости от языка файла),
  \term{автодополнением}\ и вызовом программ-утилит нажатием сочетаний 
  клавиш. Например нажатием \keys{F3}\ в \eclipse\ можно переместится на
  определение функции, на имени которой находится текствый курсор.
  
  Автодополнение\ --- редактор предлагает варианты полного написания
  идентификаторов и ключевых слов по первым буквам и нажатию обычно
  \keys{Ctrl+Tab} или \keys{Ctrl+N}. Также автоматически расставляются
  закрывающие скобки, закрывающие операторы управляющих структур типа begin/end,
  и генерируются синтаксические элементы циклов при вводе ключевых слов
  if/for/while. Особенно удобно автодополнение при написании кода на ООП 
  языках\  --- при вводе имени класса или объекта и точки предлагается меню с
  именами данных и методов класса. 
  
  При вводе имени функции и скобки выводится всплывающее окно с подсказкой\ ---
  определение функции с типом возвращаемого значения, типом и именами
  параметров.
  
  Интерфейс IDE часто предусматривает различные вспомогательные окна,
  показывающие имена и свойства объектов, описанных в программе (переменные,
  функции, структуры,..), структуру проекта с зависимостями между файлами, блоки
  справки в зависимости от текущего выделенного элемента и т.п.
  
  Часто IDE имеет встроенный графический интерфейс для отладки программ,
  используя для этого интерфейсные библиотеки для программатора и
  специальный отладочный код, добавляемый к вашей программе при
  компиляции. Используя аппаратный модуль отладки на целевом процессоре и
  отладочный код, IDE обеспечивает отображение значений и изменений регистров
  процессора, состояние переферии, позволяет задать точки останова в программном
  коде, в т.ч. условные по значению или измениею переменных или регистров
  железа.
  При использовании ОС реального времени и системы аппаратной многозадачности
  отображается загрузка ядер, загрузка процессора и используемые ресурсы для
  каждой задачи, работа планировщика, и т.п.
  
Для удобной работы доступно несколько бесплатных вариантов IDE, далее
рассмотрим два варианта: тяжелая суперуниверсальная среда \eclipse, и легкая 
в отношении требуемых ресурсов системы CodeLite.
  
  \item Пакет кросс-компилятора и утилит типа make, objdump,..
  
  Компилятор преобразует программы на языке программирования в \term{объектный
  код} (смесь кусочков машинного кода со служебной информацией) или в
  текст на языке ассемблера.
  
  \term{Кросс-компилятор}\ (arm-none-eabi-gcc) отличается от обычного
  компилятора тем, что генерирует код не для компьютера на котором он выполняется
  (\term{хост-система}, \verb|$HOST|), а для компьютера другой
  архитектуры\ --- \term{целевой} системы, \verb|$TARGET|.
  
  \term{Ассемблер}\ (as) преобразует человекочитаемый код программы в объектный
  код.
  
  \term{Линкер}\ (ld) объединяет несколько файлов объектного кода в один,
  и корректирует машинный код с учетом его конечного размещения в памяти
  целевой системы (адреса переменных, адреса переходов, размещение сегментов
  кода и данных в физической памяти целевой системы).
  
  \term{Дампер}\ (objdump) преобразует сегменты кода/данных из файла,
  полученного линкером, в формат, необходимый для ПО программатора: бинарные файлы, Intel
  HEX, ELF,.. загружаемые в масочное ПЗУ, FlashPROM (и EEPROM данных на МК
  ATmega).
  
  \item ПО для программатора, JTAG-адаптера
  
  Загрузка полученной прошивки в целевое устройство, редактирование памяти, 
  внутрисхемная отладка в процессе работы устройства, прямое измение сигналов на
  выводах процессора (граничное сканирование и тестирование железа).
  
  \item Система верстки документации
  
  Для документирования проектов и написания руководств нужна система верстки
  документации, выполняющая трансляцию текстов программ и файлов 
  документации в выходной формат, чаще всего \file{.pdf} и \file{.html}.
  
  Необходимо использовать человеко-читаемые простые текстовые файлы 
  (\file{plain ascii text}, кодировка по выбору, удобнее всего \file{utf8}) и
  использовать язык разметки\ --- DocBook, а удобнее всего \LaTeX.
  
  \alarm{
  Ни в коем случае не используйте для документации всякую бинарщину тип
  NarcoSoft Word\ --- текстовый формат необходим для корректной 
  и полноценной работы VCS. 
  Исключение по необходимости\ --- только графические файлы, подключаемые
  при генерации выходных файлов документации.
  }
  
  Эта книга написана с использованием языка разметки \LaTeX, и транслируется
  в экранный \file{.pdf} с помощью пакета \win{MiKTeX}/\lin{TeXlive}.
  
  Установка описана в \labref{texinstall}
  
\end{itemize}

