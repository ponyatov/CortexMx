\section{\copyright}
\cp{двойной лось
\url{http://www.doulos.com/knowhow/arm/CMSIS/CMSIS\_Doulos\_Tutorial.pdf}}

\bigskip
This tutorial material is part of a series to be published progressively by
Doulos.

\ru{Этот учебный материал является частью регулярных публикаций ДвухЛосей}

\bigskip
You can find the full set of currently published Tutorials and register for
notification of future additional at \url{www.doulos.com/knowhow}

\ru{Вы можете найти полный набор опубликованных методичек и зарегестироваться
для получения оповещений о новых выпусках на \url{www.doulos.com/knowhow}.}

\bigskip
You can also download the full source code of the examples used within the
Tutorial at the same URL.

\ru{По тому же URLу вы также можете скачать полную версию исходных кодов
примеров.}

\bigskip
Also check out the Doulos ARM Training and service options at
\url{www.doulos.com/arm}

\ru{Также посмотрите варианты учебных курсов Duolos по ARMам на
\url{www.doulos.com/arm}.}

\bigskip
Or email \email{info@doulos.com} for further information

\ru{Или запросите дополнительную информацию по \email{info@doulos.com}}

\bigskip
First published by Doulos March 2009

\ru{Первая публикация от Duolos Март 2009}

\bigskip
Copyright 2009 Doulos. All rights reserved. All trademarks acknowledged. All
information is provided “as is” without warranty of any kind.

\section{Введение}

The Cortex Microcontroller Software Interface Standard (CMSIS) supports
developers and vendors in creating reusable software components for 
\arm\ \cm{}\ based systems.

\ru{
Cortex Microcontroller Software Interface Standard (CMSIS)\footnote{стандарт
программного интерфейса микроконтроллеров Cortex}\ обеспечивает разработчикам и
производителям МК создание повторно используемых программных компонентов для
систем на основе микроконтроллеров \cm{}.
}

The \arm\ \cm{3}\ processor is the first core from ARM specifically designed for
the Microcontroller market. This core includes many common features (NVIC,
Timer, Debug-hardware) needed for this market. This will enable developers to
port and reuse software (e.g. a real time kernel) with much less effort to
\cm{3}\ based MCUs.

\ru{
Процессор \arm\ \cm{3}\ первое ядро от компании \arm\ специально разработанное
для рынка микроконтроллеров. Это ядро включает множество типовых блоков (NVIC,
таймеры, отладочный интрефейс) необходимых на этом рынке. Это позволяет
разработчикам с минимальными усилиями портировать и повторно использовать уже
написанное ПО\footnote{например ядро ОС реального времени}\ для МК семейства
\cm{3}\ любых производителей.
}

With a significant amount of hardware components being identical, a large
portion of the Hardware Abstraction Layer (HAL) can be identical. However,
reality has shown that lacking a common standard we find a variety of HAL/driver
libraries for different devices, which, as far as the \cm{3}\ part is
concerned essentially do the same thing\ --– just differently.

\ru{
Благодаря идентичности большого колиства аппаратных компонентов, также
идентичным оказывается и Hardware Abstraction Layer (HAL)\footnote{программный
слой аппаратной абстракции}.
Тем не менее, реальность показывает что отсутствие общего стандарта приводит к
множеству несовместимых версий библотек HAL и драйверов для различных МК, что не
соответствует идее полной переносимости ПО в серии \cm{3}.
}

The latest study of the development for the embedded market shows that software
complexity and cost will increase over time, see figure left.
Reusing Software and having a common standard to govern how to write and debug
the software will be essential to minimising costs for future developments.

\ru{
Последние исследования разработок для рынка встраиваемого ПО показывают, что
сложность программного обеспечения и его стоимость постоянно увеличиваются.
Повторное использование кода и наличие общего стандарта управляющего способами
написания и отладки ПО необходимы для минимизации стоимости разработки и
сопровождения.

\includegraphics[height=30ex]{duolos_CMSIS/cmsisdevcosts.png}
стоимости разработки
}

\bigskip
With more \cm{3}\ based MCUs about to come onto the market, ARM has recognized
that after solving the diversity issue on the hardware side, there is still a
need to create a standard to access these hardware components.

\ru{
Анализируя ситуацию с взрывным ростом количества моделей МК \cm{3}\ на рынке,
компания \arm\ обнаружила что полная идентичность аппаратной части недостаточна
для обеспечения совместимости, и необходимо создание стандарта доступа к
аппаратным компонентам.
}

The result of that effort is CMSIS; a framework to be extended by vendors, while
taking advantage of a common API (Application Programming Interface) for core
specific components and conventions that define how the device specific portions
should be implemented to make developers feel right at home when they reuse code
or develop new code for ARM \cm{}\ based devices.

\ru{
Результатом этих исследований является \cmsis: фреймворк, расширямый
поставщиками МК, с сохранением полезных свойств общего API (Application
Programming Interface)\footnote{прикладной программный интерфейс} для ядреных
компонентов и соглашениями о том, как должны быть реализованы части зависимые от
железа, чтобы разработчики чувствовали себя как дома при повторном использовании
ил разработки нового кода для семейства \cm{}.
}

\section{Структура \cmsis}

CMSIS can be divided into three basic function layers:

\ru{\cmsis\ может быть поделен на три основных слоя:}

\begin{itemize}
\item Core Peripheral Access Layer (CPAL)

The lowest level defines addresses, and access methods for common components and
functionality that exists in every Cortex-M system. Access to core registers,
NVIC, debug subsystem is provided by this layer. Tool specific access to special
purpose registers (e.g. CONTROL, xPSR), will be provided in the form of inline
functions or compiler intrinsics. This layer will be provided by ARM.

\ru{
Самый нижний уровень определяет адреса, и методы доступа к общим компонентам и
функциям, существующим в каждой \cm{}-системе. Этим уровнем
описывается доступ к регистрам ядра, NVIC\footnote{Nested Vector Interrupt
Controller, контроллер вложенных прерываний}, подсистеме отладки.
Инструментальный доступ к спецрегистрам (\file{CONTROL},\file{xPSR})
предоставляется в форме inline-функций или интринсик компилятора. Этот уровень
обеспечивается лицензиатом архитектуры\ --- компанией \arm.
}

\item Middleware Access Layer (MWAL)

This layer is also defined by ARM, but will be adapted by silicon vendors for
their respective devices.
The Middleware Access Layer defines a common API for accessing peripherals. The
Middleware Access Layer is still under development and no further information is
available at this point.

\ru{
Этот слой также специфицируется \arm, но адаптируется производителем кристаллов
для их конкретных изделий. Слой MWAL определяет общий API для доступа к
периферии. Этот слой все еще находится на стадии доработки, и на текущий момент
более подробная информация неступна.
}

\item Device Peripheral Access Layer (DPAL)

Hardware register addresses and other definitions, as well as device specific
access functions will be defined in this layer. The Device Peripheral Access
Layer is very similar to the Core Peripheral Access Layer and will be provided
by the silicon vendor. Access methods provided by CPAL may be referenced and the
vector table will be adapted to include device specific exception handler
address.

\ru{
Слой содержит адреса аппаратных регистров и другие определения, в том числе
функции доступа к специфичным особенностям чипов.
DPAL сильно похож на CPAL, но предоставляется поставщиком кристаллов.
В CPAL могут быть описаны методы доступа и адаптированная таблица векторов,
содержащая обработчики исключений, специфичные для конкретного МК.
}

While DPAL is intended to be extended by the silicon vendor, let’s not forget
about Cortex-M based FPGA products, which effectively put developers into the
position of a silicon vendor.

\ru{
DPAL предназначен для расширения вендором, но не стоит забывать о
FPGA-продуктах с примением \cm{}-ядер, которые ставят разработчиков в положение
вендора.
}

\end{itemize}

The basic structure and the functional flow is illustrated in the Figure 2. below.

\bigskip
\includegraphics[width=0.9\textwidth]{duolos_CMSIS/cmsisstruc.png}

Figure 2 CMSIS Structure functional flow

\ru{Рис.2 Функциональная структура CMSIS}

\bigskip
As far as MCU based systems are concerned it might make sense for developers to
treat the entire PCB system as monolithic block. There is no reason to
differentiate between a memory mapped register inside the MCU and a memory
mapped register external to the MCU, connected via external memory interface.
The benefit of applying a standard like CMSIS is that existing guidelines on how
to access these devices set a clear goal on how to implement and integrate
critical parts of the software.
Other team members will find a familiar environment.

\subsection{Структура файлов}

File names in CMSIS are standardized as follows:

\begin{tcolorbox}
\begin{tabular}{l l}
\file{core\_cm3.h} &\cm{3}\ global declarations and definitions, static
function definitions \\
\file{core\_cm3.c} &\cm{3}\ global definitions\\
\file{<device>.h} &Top-level header file (device specific). To be included by
application code.\\
&Includes \file{core\_cm3.h} and \file{system\_<device>.h}\\
\file{system\_<device>.h} &Device specific declarations\\
\file{system\_<device>.c} &Device specific definitions, e.g.
\verb|SystemInit()|\\
\end{tabular}
\end{tcolorbox}

Application code will only include the top-level header file which implicitly
pulls in all other essential

\bigskip
\includegraphics[width=0.5\textwidth]{duolos_CMSIS/stm32.png}
\bigskip

header files. The illustration below shows the flow and dependencies of the
header files \file{stm32.h}, \file{core\_cm3.h} and \file{system\_stm32.h},
which are part of CMSIS release V1P0.

\begin{lstlisting}[style=cpp]
/* Configuration of the Cortex-M3 Processor and Core Peripherals */
#define __MPU_PRESENT 0			 /*!< STM32 does not provide a MPU present or not*/
#define __NVIC_PRIO_BITS 4		 /*!< STM32 uses 4 Bits for the Priority Levels */
#define __Vendor_SysTickConfig 0 /*!< Set to 1 if different SysTick Config isused */
 
#include "core_cm3.h" 	 		 /* Cortex-M3 processor and core peripherals */ 
#include "system_stm32.h" 		 /* STM32 System */
\end{lstlisting}

The \file{<device>.h} file is the central include file and provided by the
silicon vendor. The application programmer is using that as the main include file in his
C source code. Note that the ARM \cm{3}\ has some optional hardware features
(e.g. the MPU, number of Interrupts and the number of the NVIC priority bits)
the silicon vendors may have implemented differently. The listing above shows
that STM32 implements four out of eight possible priority bits. The macro
\verb"__NVIC_PRIO_BITS"\ is set here to 4. STM32 does not offer a Memory
Protection Unit (MPU). Accordingly, the macro \verb"__MPU_PRESENT"\ has the
value 0.

The next example shows the corresponding definitions for a NXP LPC17xx device.
In this Cortex-M3 implementation five priority bits have been implemented and an
MPU is available.

\begin{lstlisting}[style=cpp]
/* Configuration of the Cortex-M3 Processor and Core Peripherals */
#define __MPU_PRESENT 1 		 /*!< MPU present or not */
#define __NVIC_PRIO_BITS 5 		 /*!< Number of Bits used for Priority Levels */
#define __Vendor_SysTickConfig 0 /*!< Set to 1 if different SysTick Config is used */

#include "..\core_cm3.h" 	/* Cortex-M3 processor and core peripherals */
#include "system_LPC17xx.h" /* System Header */
\end{lstlisting}

The \verb"__Vendor_SysTickConfig"\ defined is showing in both cases the default
setting. When this macro is set to 1, the \verb|SysTickConfig()|\ function in
the \file{cm3\_core.h}\ is excluded. In this case the file 
\file{<device>.h}\ must contain a vendor specific implementation of this
function.

\subsection{Независимость от тулчейна}

CMSIS exists in a three-dimensional space of the form
vendor$\div$device$\div$tool chain.
In order to remove one dimension (tool chain), the common files
\file{core\_cm3.c}\ and \file{core\_cm3.h}\ contain all essential tool specific
declarations and definitions.

\bigskip
Example:

\begin{lstlisting}[style=cpp]
/* define compiler specific symbols */
#if defined ( __CC_ARM )
	#define __ASM __asm				/*!< asm keyword for armcc */
	#define __INLINE __inline		/*!< inline keyword for armcc */
#elif defined ( __ICCARM__ )
	#define __ASM __asm				/*!< asm keyword for iarcc */
	#define __INLINE inline			/*!< inline keyword for iarcc.
									Only avaiable in High optimization mode! */
	#define __nop __no_operation	/*!< no operation intrinsic in iarcc */
#elif defined ( __GNUC__ )
	#define __ASM asm 				/*!< asm keyword for gcc */
	#define __INLINE inline 		/*!< inline keyword for gcc */
#endif
\end{lstlisting}

The remaining parts of CMSIS can now simply use the macro \verb|__INLINE|\ to
define an inline function.

\ru{Остальная часть \cmsis\ теперь может просто использовать макрос}
\verb|__INLINE|\ \ru{для определения инлайн-функций.}

\bigskip
Currently three of the most important C-compilers are supported: ARM RealView
(armcc), IAR EWARM (iccarm), and GNU Compiler Collection (gcc). This is expected
to cover the majority of tool chains.

\ru{
На настоящий момент поддерживаются три наиболее применяемых Си-компилятора:
ARM RealView (armcc), IAR EWARM (iccarm), и GNU Compiler Collection (gcc).
}

\subsection{Стандарт безопасности ПО MISRA-C}

Besides defining an API for Cortex-M core peripherals and guidelines on how to
support device peripherals, CMSIS defines some coding guidelines and
conventions. Most important is that the CMSIS code base is MISRA-C 2004
compliant, which implies that every extension should be compliant, too. MISRA-C
is a set of safety rules established by the “Motor Industry Software Reliability
Association” for the C programming language. Maintaining MISRA compliance can be
tricky, in particular when implementing driver level software. Therefore,
pragma-like exceptions in PCLint style are scattered across the source code. Be
aware that other tools, e.g. MISRA checker in IAR EWARM, might flag errors. Each
exception is accompanied with a comment explaining why this exception was made.

\subsection{Функции CPAL}

All functions in the Core Peripheral Access Layer are reentrant and can be
called from different interrupt service routines (ISR). CPAL functions are also
non-blocking\footnote{Memory barriers are exempt from that rule although they
might stall the processor for a few cycles.}\ in the sense that they do not 
contain any wait-loops.

The majority of functions in the CPAL have been implemented in the header file
\file{core\_cm3.h}\ as \verb|static| \verb|inline| functions. This allows the
compiler to optimize the function calls by placing the instructions that make up the called
function along with other code from which the function was called.

\subsection{Подпрограммы обработки прерываний}

Exception handlers will get a name suffix \verb|_Handler|, while (external)
interrupt handlers get the suffix \verb|_IRQHandler|. There must be a default
handler for each interrupt, which executes an infinite loop.
Tool specific configuration must make sure that this default handler will be
used as fall-back if no user-provided handler exists. It done
Through \verb|__weak| declaration in EWARM and RVCT armcc,
\verb|__attribute__((weak))| in GCC and RVCT armcc and
\verb|[WEAK]| export in RVCT/armasm.

Given that the \cm{}\ NVIC provides byte-arrays and bit-strings to configure
priorities and interrupt source en-/disable, an enumerated type \verb|IRQn_t|
with an element for each exception/interrupt position with the suffix
\verb|_IRQn|\ must be defined for each interrupt (\file{<device>.h}). The system
handler names are common for all devices and must not be changed.

\begin{lstlisting}[style=cpp,title=Listing shows the generic part of the
(\file{<device>.h}) file.] typedef enum IRQn
{
/****** Cortex-M3 Processor Exceptions Numbers *******************************/
 NonMaskableInt_IRQn = -14, 	/*!< 2 Non Maskable Interrupt */
 MemoryManagement_IRQn = -12, 	/*!< 4 Cortex-M3 Memory Mgmt Interrupt */
 BusFault_IRQn = -11, 			/*!< 5 Cortex-M3 Bus Fault Interrupt */
 UsageFault_IRQn = -10, 		/*!< 6 Cortex-M3 Usage Fault Interrupt */
 SVCall_IRQn = -5, 				/*!< 11 Cortex-M3 SV Call Interrupt */
 DebugMonitor_IRQn = -4, 		/*!< 12 Cortex-M3 Debug Monitor Interrupt */
 PendSV_IRQn = -2, 				/*!< 14 Cortex-M3 Pend SV Interrupt */
 SysTick_IRQn = -1, 			/*!< 15 Cortex-M3 System Tick Interrupt */
/****** Device specific Interrupt Numbers ************************************/
 UART_IRQn = 0, 				/*!< Example Interrupt */
} IRQn_Type;
\end{lstlisting}

All system handlers have negative virtual slot numbers so that they can be
distinguished in functions that abstract from the differences between system
handlers and external interrupt handlers. External interrupt handlers start at
the index 0.

\subsection{Другие соглашения о коде}

The CMSIS documentation recommends a few more things regarding capitalization of
identifiers, commenting code.

\subsubsection{Идентификаторы}

\begin{itemize}
  \item Capital names to identify Core Registers, Peripheral Registers, and CPU
  Instructions.
  
  E.g.: \verb|NVIC->AIRCR|, \verb|GPIOB|, \verb|LDMIAEQ|
  
  \item “CamelCase” (mix of upper- and lower-case letters) names to identify
  peripherals access functions and interrupts.

 E.g.: \verb|SysTickConfig()|, \verb|DebugMonitor_IRQn|

  \item Peripheral prefix (\verb|<name>_|) to identify functions that belong to
  specific peripherals.

 E.g.: \verb|ITM_SendChar()|, \verb|NVIC_SystemReset()|

\end{itemize}

\subsubsection{Комментарии}

CMSIS uses \term{Doxygen}\ style comments for all definitions and encourages
developers to do the same.

In particular, the comment for each function definition should at least contain

\begin{itemize}
\item one-line brief function overview. (Tag: \verb|@brief|)
\item detailed parameter explanation. (Tag: \verb|@param|)
\item detailed information about return values. (Tag: \verb|@return|)
\item detailed description of the actual function.
\end{itemize}

The example below shows the beginning of a function definition:

\begin{lstlisting}[style=cpp]
/**
 * @brief Enable Interrupt in NVIC Interrupt Controller
 *
 * @param IRQn_Type IRQn specifies the interrupt number
 * @return none
 *
 * Enable a device specific interupt in the NVIC interrupt controller.
 * The interrupt number cannot be a negative value.
 */
static __INLINE void NVIC_EnableIRQ(IRQn_Type IRQn)
{
// . . .
}
\end{lstlisting}

The tags can be parsed by the documentation tool Doxygen, which is used to
create cross-referenced source code documentation in various formats
(\url{http://www.stack.nl/~dimitri/doxygen/index.html}). The tag syntax is
rather minimalistic and does not impair readability of the source code. Please
consult the Doxygen Documentation for details about tag syntax.

As an alternative to regular C block comments (\verb|/* */|) CMSIS explicitly
allows line comments (\verb|//|, so called C++-comments). If you are concerned
about MISRA compliance, be aware though, that MISRA-C 2004 doesn’t allow line
comments according to rule 2.2.

\ru{
CMSIS предлагает альтернативу обычным сишным /* блочным комментариям */ в виде
// строчных комментариев, так называмых \cpp-комментариев. Если вас беспокоит
выполнение превил MISRA, учтите что} \alarm{MISRA-C:2004 не разрешает
использовать строчные комментарии согласно правилу 2.2}.

\subsubsection{Типы данных}

All data types referenced by CMSIS are based on those defined in the standard C
header file \file{stdint.h}. Data structures for core registers are defined
CMSIS header file \file{core\_cm3.h}, along with macros for qualifying registers
according to their access permissions. The rationale is that tools might be able
to automatically extract that information for debug purposes.

\begin{lstlisting}[style=cpp]
#define 	__I 	volatile const	/*!< defines 'read only' permissions */
#define		__O 	volatile 		/*!< defines 'write only' permissions */
#define 	__IO 	volatile 		/*!< defines 'read / write' permissions */
\end{lstlisting}

\subsection{Отладка}

A common requirement in software development is some sort of terminal output for
debugging. Text/graphics displays in embedded devices cannot be assumed to be at
hand (or might be in use), which previously left the developer with essentially 
two choices:

\begin{enumerate}
\item Use one of the ubiquitous UARTs and connect a terminal

Issues: All UARTs might be in use, access to UART signals might not be possible
for reasons that include pin-sharing, PCB layout, etc.

\item Use the semihosting mechanism

Issue: Significant software overhead on target CPU, might not be supported in
the same way by all tool chains, potential impact on timing behavior.
\end{enumerate}

With \cm{3}\ the preferred method makes use of the Instrumentation Trace
Macrocell (ITM), which is part of the processor macro cell and thus always
present.\footnote{Always present in \cm{3}\ rev1 cores. \cm{3}\ rev2 makes
ITM an optional feature.}

A Serial Wire Viewer (SWV) capable debug adapter can receive ITM data through
the SWO (Serial Wire Out) debug pin. ITM implements 32 general purpose data
channels. CMSIS builds on top of this and declares channel 0 to be used for
terminal output, along with a function called \verb|ITM_SendChar()| which can be
used as low-level driver function for printf-style output. A second channel (31) 
has been reserved for OS aware debugging, which means that a kernel can use it
to transmit kernel specific data which could then be interpreted by the debug
tool. 

With this standardization, tool vendors have it much easier to implement
specific debug features, such as e.g. terminal emulation for data received via
ITM channel 0. Developers on the other hand can rely on this feature to dump
state information, without having to configure UARTs and external terminal
emulators. See our Tutorial 2 later in this document.

Access privilege can be configured for groups of ITM channels. In order to use
ITM channel 0, unprivileged access must be granted, whereas ITM channel 31 is in
a different group and may allow privileged access only.

\subsection{Контроль версии CMSIS}

The CMSIS developers have taken care to provide macros indicating the CMSIS
version used in a project. That way provisions can be made to prevent code to be
used with a different CMSIS version than originally intended.

\ru{
Разработчики \cmsis\ позаботились о предоставлении макросов версии \cmsis,
использованной в проекте. Это способ для защиты от использования кода,
предназначенного для другой версии \cmsis.
}

\begin{lstlisting}[style=cpp]
#define __CM3_CMSIS_VERSION_MAIN (0x01)		/* [31:16] main version */
#define __CM3_CMSIS_VERSION_SUB (0x10)		/* [15:0] sub version */

#define __CM3_CMSIS_VERSION ((__CM3_CMSIS_VERSION_MAIN << 16) | \
							__CM3_CMSIS_VERSION_SUB)
\end{lstlisting}

\section{Урок 1 – Первый пример}

In order to explain application of CMSIS in real projects, we are going to look
at a simple example of a \cm{3}\ application. The program compiles for STM32
processors and a project file for the Keil $\mu$Vision IDE has been provided.
Porting the example to other tool chains, such as IAR EWARM is straight forward
and the IAR EWARM version is provided as well. A great number of CMSIS function
definitions can be found in \file{core\_cm3.h}\ as “static inline” functions.
Depending on the compiler optimization level, this helps getting very efficient
instruction sequences rather than actual function calls, while ensuring a
certain level of type safety.

After clock and GPIO initialization, the SysTick timer is configured to a period
of 0.5 seconds. Whenever the handler executes it toggles the state of
\verb|GPIOB[15]|.

\subsection{Пример 1 Точка входа}

Initially, the program was implemented using STMicroelectronics’ FWLib, a
library that provides access to \cm{3}\ internals and STM32 peripherals. Near
to medium term, firmware libraries such as FWLib will be based on CMSIS. Parts
of FWLib that will eventually form the DPAL (see above).

\lstinputlisting[inputencoding=cp1251,style=cpp]{duolos_CMSIS/stm1.c}

\lstinputlisting[inputencoding=cp1251,style=cpp]{duolos_CMSIS/stm2.c}

The two listings above show the contents of main.c, and \file{stm32f10x\_it.c}.
This latter file contains interrupt handler templates from ST’s FWLib, which
have to be adapted to implement project specific functionality.

\subsection{Пример 2 Адаптация для CMSIS}

In a second step, the program has been converted to using CMSIS. The CMSIS
version used is V1P10 as downloaded via the link above. We want to make sure to
use the same CMSIS that has been used to develop the program and check the
version number.

\lstinputlisting[inputencoding=cp1251,style=cpp]{duolos_CMSIS/stm3.c}

Initial support for STM32 MCU is part of CMSIS and is pulled in by including the
header file \file{stm32.h}. At the point of writing this tutorial a fully CMSIS
complaint FWLib was not available so some compromises and hand adjustments hand
to be made. Fore this reason, we will include both, FWLib and CMSIS files. Until
vendors have full adopted CMSIS small issues will have to be dealt with when
combining CMSIS with a vendor library.

In this case simply including the FWLib main header file \file{stm32f10x\_lib.h}
in addition to \file{stm32.h} triggers a number of error messages caused by
multiple definitions of functions and macros. To avoid this, we will have to
selectively include individual FWLib headers (see below). All FWLib headers
depend on definitions in the files \file{cortexm3\_core.h} and
\file{stm32f10x\_map.h}. Most of the definitions in these two header files have
already been defined by CMSIS (\file{core\_cm3.h} and \file{system\_stm32.h})
and we have to pretend to FWLib that both header files had been included already.

\begin{lstlisting}[style=cpp]
// Prevent interference with FWLib
#define __STM32F10x_MAP_H
#include <stm32f10x_type.h>
#include <stm32f10x_gpio.h>
#include <stm32f10x_rcc.h>
\end{lstlisting}

Actual system initialization will be encapsulated by the CMSIS function
\verb|SystemInit()|, which has to be implemented by the silicon vendor. As a
minimal requirement, this function would initialize the MCU’s clock system. In
case of the reference implementation in \file{system\_stm32.c},
\verb|SystemInit()| also initializes the Flash memory interface. CMSIS defines a
single system variable, \verb|SystemFrequency|, which is supposed to reflect the
frequency of both core and SysTick timer in Hz. This concept is sufficient for a
minimal implementation but will likely have to be extended for actual MCU as
demonstrated by CMSIS’ \file{system\_stm32.c}, in which several variables have
been defined to hold the frequency values of different clock domains in the
STM32 MCU. SysTick timer and core could have different frequencies and care must
be taken when using SystemFrequency in a program.

Current CMSIS does not initialize peripheral clocks and it is arguable whether
it should. In any case we use the corresponding FWLib function to enable GPIOB
clock.

\begin{lstlisting}[style=cpp]
// Initialization moved to SystemInit() in system_stm32.c. Clock
// configuration now handled by #defines. Use uVision
// configuration wizard or text editor to change.
SystemInit(); 										// *** CMSIS change ***
// APB peripherals still have to be enabled individually.
RCC_APB2PeriphClockCmd(RCC_APB2Periph_GPIOB, ENABLE);
\end{lstlisting}

NVIC group- and sub-priority configuration is handled by the function
\verb|NVIC_SetPriorityGrouping()|, where the direct encoding of the
\verb|PRIGROUP| field in \verb|SCB->AIRCR| is used. We choose the value 4, which
represents 3 bits for group (preempting) and 5 bits for the sub priority. A
general formula to calculate the proper value is \verb|PRIGROUP = bitssub-1|.

\begin{lstlisting}[style=cpp]
// priority configuration: 3.5
NVIC_SetPriorityGrouping(4); 		// *** CMSIS change ***
\end{lstlisting}

Different from the initial version of the example, the CMSIS variant does not at
this point set the SysTick handler priority. This is part of the SysTick
initialization and will be covered later.

\begin{lstlisting}[style=cpp]
GPIO_Init(GPIOB, &GPIOB_InitStruct);
GPIO_WriteBit(GPIOB, GPIO_Pin_All, Bit_RESET);
\end{lstlisting}

Following NVIC set-up, we use plain FWLib functions to configure GPIO port B.

\begin{lstlisting}[style=cpp]
SysTick_Config(SystemFrequency/2); // *** CMSIS Change ***

// SysTick_Config() hardcodes priority. We will overwrite this.
NVIC_SetPriority(SysTick_IRQn, 14); // *** CMSIS change ***
\end{lstlisting}

\verb|SysTick_Config()|, provided by CMSIS, programs the reload register with
the parameter value. The function also selects HCLK (core clock) as clock
source, enables SysTick interrupts and starts the counter. The function also
fixes the SysTick handler priority to the lowest priority in the system, which
is the recommended SysTick priority for use in an RTOS scheduler for instance.
In our example, however, we prefer a different priority and override the hard
coded value with an additional call to \verb|NVIC_SetPriority()|. This function
abstracts from the difference between \cm{3}\ system handlers and external
interrupt handlers. All configurable system exceptions will be identified by
negative IRQ numbers (see above).

\begin{lstlisting}[style=cpp]
__irq void SysTick_Handler()
{
	static BitAction toggle = Bit_SET;
	GPIO_WriteBit(GPIOB, GPIO_Pin_15, toggle);
	if (toggle == Bit_SET) {
		puts("Pin state is ON");
		toggle = Bit_RESET;
	} else {
		puts("Pin state is OFF");
		toggle = Bit_SET;
	}
}
\end{lstlisting}

The SysTick handler code above does not need any modification. FWLib naming
conventions is complying with CMSIS, in that the names of all internal exception
handlers must end in \verb|_Handler|. Names of external interrupt handlers must
end in \verb|_IRQHandler|. The handler implementation accesses the GPIO port via
FWLib functions and definitions.

\section{Урок 2 – ITM Отладка}

To exercise some CMSIS debug functionality for our first example from tutorial
1, we add debug output messages via calls to \verb|puts()|. We will now redirect
character output to Instrumentation Trace Macrocell (ITM), (remember that CMSIS
reserves ITM channel 0 for this), using the CMSIS function
\verb|ITM_SendChar()|. A mechanism called retargeting enables us to provide our
own implementation of a system function.

\begin{lstlisting}[style=cpp]
// retarget fputc() for debug output via ITM
int fputc(int c, FILE *stream) {
	return (int)ITM_SendChar((uint32_t)c);
}
\end{lstlisting}

The standard C function \verb|fputc()|, which will eventually be called by
\verb|puts()| in our SysTick handler, will be re-implemented, taking advantage
of the function \verb|ITM_SendChar()|. The result of this retarget can now be
easily monitored in $\mu$Vision ITM viewer as shown in the screenshot below.

\includegraphics[width=0.5\textwidth]{duolos_CMSIS/ITMshot.png}

If you are going to use the IAR EWARM tool for this example it is not necessary
to manually retarget this function. Instead, in the project options dialog under
“General Options” there is a tab “Library Configuration” tab, which offers
checkboxes to enable this functionality. The screenshot below shows which
settings are required to redirect standard output.

\includegraphics[width=0.6\textwidth]{duolos_CMSIS/simplecmsis.png}

\section{Заключение}

The CMSIS will reduce the learning-curve for the application programmers by
providing a consistent software framework, ensuring consistent documentation and
easy deployment of boilerplate code across various compiler vendors. The
consequent use and implementation of CMSIS across many silicon and middleware
software partners will simplify the verification and certification process and
therefore reduce future project risk. The adapted common programming techniques
though CMSIS will simplify the long term maintenance due to easier to understand
source code. The silicon vendors can focus on there added value and device
features. All reasons together will reduce software development cost and time to
get new products to the market.
