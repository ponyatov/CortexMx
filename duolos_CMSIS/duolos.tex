\documentclass[oneside,14pt]{book}

\usepackage[T1,T2A]{fontenc}
\usepackage[utf8]{inputenc}
\usepackage[english,russian]{babel}
\usepackage{indentfirst}

%\usepackage[paperwidth=15cm,paperheight=7.5cm]{geometry} % планшет
%\usepackage[paperwidth=297mm,paperheight=210mm]{geometry} % A4
%\usepackage[paperwidth=148mm,paperheight=105mm]{geometry} % A5
\usepackage[paperwidth=18cm,paperheight=13cm,margin=5mm]{geometry} % экран*2

%\usepackage[colorlinks=true,
%]{hyperref}
%\newcommand{\email}[2]{#1\ \href{mailto:#2}{<\nolinkurl{#2}>}}
% %\newcommand{\email}[2]{\emph{#1\ <#2>}}
\usepackage[unicode,colorlinks,
pdftitle={Azbuka ARMaturschika (ru)},
pdfauthor={(c) Dmitry Ponyatov <dponyatov@gmail.com>, SSAU ASCL},
pdfsubject={ru manual on writing programs for Cortex-M MCUs},
pdfkeywords={ARM} {Cortex} {MCU} {ARMatura} {Arduino} {SSAU}
]{hyperref}

\usepackage{wrapfig}
\usepackage{graphicx}
\usepackage{epstopdf}
\DeclareGraphicsExtensions{.eps}

\usepackage{listings}
\usepackage{dirtree}
\usepackage[usenames,dvipsnames,svgnames]{xcolor}
\newcommand{\cppcolor}{\color[rgb]{0.94, 0.97, 1.0}} % Alice blue
\newcommand{\asmcolor}{\color[rgb]{0.98, 0.92, 0.84}} % Antique white
\newcommand{\concolor}{\color[rgb]{0.88, 1.0, 1.0}} % Light cyan

%\definecolor{cppcolor}{rgb}{0.94, 0.97, 1.0}
\lstset{frame=single,
numbers=left, numberstyle=\small,
commentstyle=\color{cyan}\texttt,
tabsize=4,
keywordstyle=\color{blue}\texttt
}
\usepackage{lstlangarm}

\lstdefinestyle{cpp}{language=C++,backgroundcolor=\cppcolor}
\lstdefinestyle{asm}{language={[ARM]Assembler},backgroundcolor=\asmcolor}
\lstdefinestyle{con}{backgroundcolor=\concolor}

\newcommand{\cm}[1]{Cortex-M#1}
\newcommand{\cx}{\cm{x}}

\newcommand{\vld}{STM32VLDISCOVERY}

%\renewcommand{\url}[1]{\textbf{#1}}
\newcommand{\email}[1]{$<$\href{mailto:#1}{\textbf{#1}}$>$}

\newcommand{\cpp}{$C^{+^{+}}$}

\newcommand{\cp}[1]{\footnote{копипаста: #1}}

\newcommand{\thmod}{Thumb}
\newcommand{\armod}{ARM}

\newcommand{\Reg}[1]{\textbf{#1}}
\newcommand{\R}[1]{\Reg{R#1}}

\newcommand{\periph}[1]{\texttt{#1}}
\newcommand{\jtag}{\periph{JTAG}}

\usepackage{wasysym} % smileys
\usepackage{gensymb} % celsius
\usepackage{amssymb} % windows key
\usepackage{textcomp} % bigcircle

\usepackage[os=win]{menukeys}
\newcommand{\winstart}{$\boxplus$}
\newcommand{\file}[1]{\textbf{\textsf{#1}}}
\newcommand{\window}[1]{\textbf{\textit{#1}}}
\newcommand{\alarm}[1]{{\color{DarkRed}#1}}
\newcommand{\wcmd}[1]{\keys{\winstart+R}\ \directory{#1}}
\newcommand{\checkbox}{$\boxtimes$}
\newcommand{\uncheckbox}{$\square$}

\newcommand{\win}[1]{\includegraphics[height=10ex]{fig/winlogo.jpg} #1}
\newcommand{\lin}[1]{\includegraphics[height=10ex]{fig/linuxcolor.png} #1}

\newcommand{\linux}{Linux}
\newcommand{\git}{Git}
\newcommand{\eclipse}{\textcircled{$\equiv$}\textsc{eclipse}}
%\newcommand{\term}[1]{\underline{#1}}
\newcommand{\term}[1]{\underline{\color{DarkBlue} #1}}

\usepackage{tocloft}
\newcommand{\listlabname}{Лабораторные работы}
\newlistof{lab}{ex}{\listlabname}
\newcommand{\labpart}[1]{\addcontentsline{ex}{part}{#1}}
\newcounter{labworkcounter}
\newcommand{\labwork}[1]{
\refstepcounter{labworkcounter}
\section*{ЛР\thelabworkcounter: #1}
\addcontentsline{toc}{subsection}{ЛР\thelabworkcounter: #1}
\addcontentsline{ex}{section}{ЛР\thelabworkcounter: #1}
}
\newcommand{\labref}[1]{ЛР\ref{#1}}

\newcommand{\mytitle}[1]{
\title{\Huge{Азбука халтурщика-ARMатурщика}\\
#1\\
\normalsize{учебный курс по микроконтроллерам \cx:\\
Миландр 1986ВЕ, STM32F, LPC21xx}}
}

\author{\copyright\\
Понятов Д.А. \email{dponyatov@gmail.com}, ИКП СГАУ, \\
Недяк С.П. \email{fvs@fet.tusur.ru}, ТУСУР
}


\usepackage{tcolorbox}
\newcommand{\ru}[1]{{\color{blue}#1}}

\begin{document}
\mytitle{основы CMSIS}
\maketitle
\setcounter{tocdepth}{3}\tableofcontents

\section{\copyright}
\cp{двойной лось
\url{http://www.doulos.com/knowhow/arm/CMSIS/CMSIS\_Doulos\_Tutorial.pdf}}

\bigskip
This tutorial material is part of a series to be published progressively by
Doulos.

\bigskip
You can find the full set of currently published Tutorials and register for
notification of future additional at \url{www.doulos.com/knowhow}

\bigskip
You can also download the full source code of the examples used within the
Tutorial at the same URL.

\bigskip
Also check out the Doulos ARM Training and service options at
\url{www.doulos.com/arm}

\bigskip
Or email \email{info@doulos.com} for further information

\bigskip
First published by Doulos March 2009

\bigskip
Copyright 2009 Doulos. All rights reserved. All trademarks acknowledged. All
information is provided “as is” without warranty of any kind.

\section{Introduction}

The Cortex Microcontroller Software Interface Standard (CMSIS) supports
developers and vendors in creating reusable software components for 
\arm\ \cm{}\ based systems.

\ru{
Cortex Microcontroller Software Interface Standard (CMSIS)\footnote{стандарт
программного интерфейса микроконтроллеров Cortex}\ обеспечивает разработчикам и
производителям МК создание повторно используемых программных компонентов для
систем на основе микроконтроллеров \cm{}.
}

The \arm\ \cm{3}\ processor is the first core from ARM specifically designed for
the Microcontroller market. This core includes many common features (NVIC,
Timer, Debug-hardware) needed for this market. This will enable developers to
port and reuse software (e.g. a real time kernel) with much less effort to
\cm{3}\ based MCUs.

\ru{
Процессор \arm\ \cm{3}\ первое ядро от компании \arm\ специально разработанное
для рынка микроконтроллеров. Это ядро включает множество типовых блоков (NVIC,
таймеры, отладочный интрефейс) необходимых на этом рынке. Это позволяет
разработчикам с минимальными усилиями портировать и повторно использовать уже
написанное ПО\footnote{например ядро ОС реального времени}\ для МК семейства
\cm{3}\ любых производителей.
}

With a significant amount of hardware components being identical, a large
portion of the Hardware Abstraction Layer (HAL) can be identical. However,
reality has shown that lacking a common standard we find a variety of HAL/driver
libraries for different devices, which, as far as the \cm{3}\ part is
concerned essentially do the same thing\ --– just differently.

\ru{
Благодаря идентичности большого колиства аппаратных компонентов, также
идентичным оказывается и Hardware Abstraction Layer (HAL)\footnote{программный
слой аппаратной абстракции}.
Тем не менее, реальность показывает что отсутствие общего стандарта приводит к
множеству несовместимых версий библотек HAL и драйверов для различных МК, что не
соответствует идее полной переносимости ПО в серии \cm{3}.
}

The latest study of the development for the embedded market shows that software
complexity and cost will increase over time, see figure left.
Reusing Software and having a common standard to govern how to write and debug
the software will be essential to minimising costs for future developments.

\ru{
Последние исследования разработок для рынка встраиваемого ПО показывают, что
сложность программного обеспечения и его стоимость постоянно увеличиваются.
Повторное использование кода и наличие общего стандарта управляющего способами
написания и отладки ПО необходимы для минимизации стоимости разработки и
сопровождения.

\includegraphics[height=30ex]{duolos_CMSIS/cmsisdevcosts.png}
стоимости разработки
}

\bigskip
With more \cm{3}\ based MCUs about to come onto the market, ARM has recognized
that after solving the diversity issue on the hardware side, there is still a
need to create a standard to access these hardware components.

\ru{
Анализируя ситуацию с взрывным ростом количества моделей МК \cm{3}\ на рынке,
компания \arm\ обнаружила что полная идентичность аппаратной части недостаточна
для обеспечения совместимости, и необходимо создание стандарта доступа к
аппаратным компонентам.
}

The result of that effort is CMSIS; a framework to be extended by vendors, while
taking advantage of a common API (Application Programming Interface) for core
specific components and conventions that define how the device specific portions
should be implemented to make developers feel right at home when they reuse code
or develop new code for ARM \cm{}\ based devices.

\ru{
Результатом этих исследований является \cmsis: фреймворк, расширямый
поставщиками МК, с сохранением полезных свойств общего API (Application
Programming Interface)\footnote{прикладной программный интерфейс} для ядреных
компонентов и соглашениями о том, как должны быть реализованы части зависимые от
железа, чтобы разработчики чувствовали себя как дома при повторном использовании
ил разработки нового кода для семейства \cm{}.
}

\section{Структура \cmsis}

CMSIS can be divided into three basic function layers:

\ru{\cmsis\ может быть поделен на три основных слоя:}

\begin{itemize}
\item Core Peripheral Access Layer (CPAL)

The lowest level defines addresses, and access methods for common components and
functionality that exists in every Cortex-M system. Access to core registers,
NVIC, debug subsystem is provided by this layer. Tool specific access to special
purpose registers (e.g. CONTROL, xPSR), will be provided in the form of inline
functions or compiler intrinsics. This layer will be provided by ARM.

\ru{
Самый нижний уровень определяет адреса, и методы доступа к общим компонентам и
функциям, существующим в каждой \cm{}-системе. Этим уровнем
описывается доступ к регистрам ядра, NVIC\footnote{Nested Vector Interrupt
Controller, контроллер вложенных прерываний}, подсистеме отладки.
Инструментальный доступ к спецрегистрам (\file{CONTROL},\file{xPSR})
предоставляется в форме inline-функций или интринсик компилятора. Этот уровень
обеспечивается лицензиатом архитектуры\ --- компанией \arm.
}

\item Middleware Access Layer (MWAL)

This layer is also defined by ARM, but will be adapted by silicon vendors for
their respective devices.
The Middleware Access Layer defines a common API for accessing peripherals. The
Middleware Access Layer is still under development and no further information is
available at this point.

\ru{
Этот слой также специфицируется \arm, но адаптируется производителем кристаллов
для их конкретных изделий. Слой MWAL определяет общий API для доступа к
периферии. Этот слой все еще находится на стадии доработки, и на текущий момент
более подробная информация неступна.
}

\item Device Peripheral Access Layer (DPAL)

Hardware register addresses and other definitions, as well as device specific
access functions will be defined in this layer. The Device Peripheral Access
Layer is very similar to the Core Peripheral Access Layer and will be provided
by the silicon vendor. Access methods provided by CPAL may be referenced and the
vector table will be adapted to include device specific exception handler
address.

\ru{
Слой содержит адреса аппаратных регистров и другие определения, в том числе
функции доступа к специфичным особенностям чипов.
DPAL сильно похож на CPAL, но предоставляется поставщиком кристаллов.
В CPAL могут быть описаны методы доступа и адаптированная таблица векторов,
содержащая обработчики исключений, специфичные для конкретного МК.
}

While DPAL is intended to be extended by the silicon vendor, let’s not forget
about Cortex-M based FPGA products, which effectively put developers into the
position of a silicon vendor.

\ru{
DPAL предназначен для расширения вендором, но не стоит забывать о
FPGA-продуктах с примением \cm{}-ядер, которые ставят разработчиков в положение
вендора.
}

\end{itemize}

The basic structure and the functional flow is illustrated in the Figure 2. below.

\bigskip
\includegraphics[width=0.9\textwidth]{duolos_CMSIS/cmsisstruc.png}

Figure 2 CMSIS Structure functional flow

\ru{Рис.2 Функциональная структура CMSIS}

\bigskip
As far as MCU based systems are concerned it might make sense for developers to
treat the entire PCB system as monolithic block. There is no reason to
differentiate between a memory mapped register inside the MCU and a memory
mapped register external to the MCU, connected via external memory interface.
The benefit of applying a standard like CMSIS is that existing guidelines on how
to access these devices set a clear goal on how to implement and integrate
critical parts of the software.
Other team members will find a familiar environment.

\subsection{File Structure}

File names in CMSIS are standardized as follows:

\begin{tcolorbox}
\begin{tabular}{l l}
\file{core\_cm3.h} &\cm{3}\ global declarations and definitions, static
function definitions \\
\file{core\_cm3.c} &\cm{3}\ global definitions\\
\file{<device>.h} &Top-level header file (device specific). To be included by
application code.\\
&Includes \file{core\_cm3.h} and \file{system\_<device>.h}\\
\file{system\_<device>.h} &Device specific declarations\\
\file{system\_<device>.c} &Device specific definitions, e.g.
\verb|SystemInit()|\\
\end{tabular}
\end{tcolorbox}

Application code will only include the top-level header file which implicitly
pulls in all other essential

\bigskip
\includegraphics[width=0.5\textwidth]{duolos_CMSIS/stm32.png}
\bigskip

header files. The illustration below shows the flow and dependencies of the
header files \file{stm32.h}, \file{core\_cm3.h} and \file{system\_stm32.h},
which are part of CMSIS release V1P0.

\begin{lstlisting}[style=cpp]
/* Configuration of the Cortex-M3 Processor and Core Peripherals */
#define __MPU_PRESENT 0			 /*!< STM32 does not provide a MPU present or not*/
#define __NVIC_PRIO_BITS 4		 /*!< STM32 uses 4 Bits for the Priority Levels */
#define __Vendor_SysTickConfig 0 /*!< Set to 1 if different SysTick Config isused */
 
#include "core_cm3.h" 	 		 /* Cortex-M3 processor and core peripherals */ 
#include "system_stm32.h" 		 /* STM32 System */
\end{lstlisting}

The \file{<device>.h} file is the central include file and provided by the
silicon vendor. The application programmer is using that as the main include file in his
C source code. Note that the ARM \cm{3}\ has some optional hardware features
(e.g. the MPU, number of Interrupts and the number of the NVIC priority bits)
the silicon vendors may have implemented differently. The listing above shows
that STM32 implements four out of eight possible priority bits. The macro
\verb"__NVIC_PRIO_BITS"\ is set here to 4. STM32 does not offer a Memory
Protection Unit (MPU). Accordingly, the macro \verb"__MPU_PRESENT"\ has the
value 0.

The next example shows the corresponding definitions for a NXP LPC17xx device.
In this Cortex-M3 implementation five priority bits have been implemented and an
MPU is available.

\begin{lstlisting}[style=cpp]
/* Configuration of the Cortex-M3 Processor and Core Peripherals */
#define __MPU_PRESENT 1 		 /*!< MPU present or not */
#define __NVIC_PRIO_BITS 5 		 /*!< Number of Bits used for Priority Levels */
#define __Vendor_SysTickConfig 0 /*!< Set to 1 if different SysTick Config is used */

#include "..\core_cm3.h" 	/* Cortex-M3 processor and core peripherals */
#include "system_LPC17xx.h" /* System Header */
\end{lstlisting}

The \verb"__Vendor_SysTickConfig"\ defined is showing in both cases the default
setting. When this macro is set to 1, the \verb|SysTickConfig()|\ function in
the \file{cm3\_core.h}\ is excluded. In this case the file 
\file{<device>.h}\ must contain a vendor specific implementation of this
function.

\subsection{Tool Independence}

CMSIS exists in a three-dimensional space of the form
vendor$\div$device$\div$tool chain.
In order to remove one dimension (tool chain), the common files
\file{core\_cm3.c}\ and \file{core\_cm3.h}\ contain all essential tool specific
declarations and definitions.

\bigskip
Example:

\begin{lstlisting}[style=cpp]
/* define compiler specific symbols */
#if defined ( __CC_ARM )
	#define __ASM __asm				/*!< asm keyword for armcc */
	#define __INLINE __inline		/*!< inline keyword for armcc */
#elif defined ( __ICCARM__ )
	#define __ASM __asm				/*!< asm keyword for iarcc */
	#define __INLINE inline			/*!< inline keyword for iarcc.
									Only avaiable in High optimization mode! */
	#define __nop __no_operation	/*!< no operation intrinsic in iarcc */
#elif defined ( __GNUC__ )
	#define __ASM asm 				/*!< asm keyword for gcc */
	#define __INLINE inline 		/*!< inline keyword for gcc */
#endif
\end{lstlisting}

The remaining parts of CMSIS can now simply use the macro \verb|__INLINE|\ to
define an inline function.

\ru{Остальная часть \cmsis\ теперь может просто использовать макрос}
\verb|__INLINE|\ \ru{для определения инлайн-функций.}

\bigskip
Currently three of the most important C-compilers are supported: ARM RealView
(armcc), IAR EWARM (iccarm), and GNU Compiler Collection (gcc). This is expected
to cover the majority of tool chains.

\ru{
На настоящий момент поддерживаются три наиболее применяемых Си-компилятора:
ARM RealView (armcc), IAR EWARM (iccarm), и GNU Compiler Collection (gcc).
}

\subsection{MISRA-C}

Besides defining an API for Cortex-M core peripherals and guidelines on how to
support device peripherals, CMSIS defines some coding guidelines and
conventions. Most important is that the CMSIS code base is MISRA-C 2004
compliant, which implies that every extension should be compliant, too. MISRA-C
is a set of safety rules established by the “Motor Industry Software Reliability
Association” for the C programming language. Maintaining MISRA compliance can be
tricky, in particular when implementing driver level software. Therefore,
pragma-like exceptions in PCLint style are scattered across the source code. Be
aware that other tools, e.g. MISRA checker in IAR EWARM, might flag errors. Each
exception is accompanied with a comment explaining why this exception was made.

\subsection{CPAL Functions}



\subsection{Interrupt Service Routines}
\subsection{Other Coding Conventions}
\subsubsection{Identifiers}
\subsubsection{Comments}
\subsubsection{Data Types}
\subsection{Debugging}
\subsection{Future Updates}
\section{Tutorial 1 – A First Example}
\subsection{Example 1 Starting point}
\subsection{Example 2 Converting to CMSIS}
\section{Tutorial 2 – ITM Debug}
\section{Summary}

\addcontentsline{toc}{part}{Литература}
\begin{thebibliography}{9}

\bibitem{leaps}{\copyright\ Quantum Leaps}

\bibitem{milandr}{\url{http://milandr.ru/} ЗАО <<ПКК Миландр>>}

\bibitem{progit}{\url{http://git-scm.com/book/ru} перевод:
Scott Chacon
\textbf{Pro Git}
}

\bibitem{habraQP}{\url{http://habrahabr.ru/post/114239/} хабра: Quantum Leaps QP
и диаграммы состояний в UML}

\bibitem{quantumleaps}{\url{http://www.state-machine.com/}
Quantum$^{\circledR}L^{e}aPs$ State Machines \& Tools}

\end{thebibliography}

\end{document}
