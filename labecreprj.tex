\labwork{Создание нового проекта в \eclipse}\label{labecreprj}

Создадим новый проект, напишем простую программу, и запустим ее в отладчике.

\bigskip\menu{\eclipse>File>New>Project>C/C++>C Project>Next}

\menu{Project name>hello}

\menu{Project type>Makefile project>Empty project}

\menu{Toolchains>Cross ARM GCC}

\menu{Finish}

\menu{This kind of project associated with C/C++ perspective>Yes}

 \bigskip В окне \eclpx\ появится пустая закладка
 проекта \file{hello}. Если вдруг там будут какие-то файлы, значит кто-то до вас уже
 создал проект, и что-то туда наляпал. В этом случе повторите создание, задав
 имя типа \file{hello<номер группы><FIO>}\ или типа того, для полной
 уверенности можно сначала посмотреть что в \file{D:/ARM}\ нет папки с таким именем.

\bigskip Нужно сразу настроить несколько свойств проекта.

\bigskip Команда-билдер для проекта\ --- задаем явно \file{make}:

\bigskip\menu{\eclpx>\file{hello}>\rms>Properties>C/C++ Build}

\menu{\uncheckbox\ Use default build command}

\menu{Build command:>\file{make}}

\menu{OK}


