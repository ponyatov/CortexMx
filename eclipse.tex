\secru{\eclipse}

Например нажатием \keys{F3}\ в \eclipse\ можно переместится на
  определение функции, на имени которой находится текствый курсор.
  
    Автодополнение\ --- редактор предлагает варианты полного написания
  идентификаторов и ключевых слов по первым буквам и нажатию обычно
  \keys{Ctrl+Tab} или \keys{Ctrl+N}. Также автоматически расставляются
  закрывающие скобки, закрывающие операторы управляющих структур типа begin/end,
  и генерируются синтаксические элементы циклов при вводе ключевых слов
  if/for/while. Особенно удобно автодополнение при написании кода на ООП 
  языках\  --- при вводе имени класса или объекта и точки предлагается меню с
  именами данных и методов класса. 
  
  При вводе имени функции и скобки выводится всплывающее окно с подсказкой\ ---
  определение функции с типом возвращаемого значения, типом и именами
  параметров.
  
  Интерфейс IDE часто предусматривает различные вспомогательные окна,
  показывающие имена и свойства объектов, описанных в программе (переменные,
  функции, структуры,..), структуру проекта с зависимостями между файлами, блоки
  справки в зависимости от текущего выделенного элемента и т.п.
  
  Часто IDE имеет встроенный графический интерфейс для отладки программ,
  используя для этого интерфейсные библиотеки для программатора и
  специальный отладочный код, добавляемый к вашей программе при
  компиляции. Используя аппаратный модуль отладки на целевом процессоре и
  отладочный код, IDE обеспечивает отображение значений и изменений регистров
  процессора, состояние переферии, позволяет задать точки останова в программном
  коде, в т.ч. условные по значению или измениею переменных или регистров
  железа.
  При использовании ОС реального времени и системы аппаратной многозадачности
  отображается загрузка ядер, загрузка процессора и используемые ресурсы для
  каждой задачи, работа планировщика, и т.п.
  
Для удобной работы доступно несколько бесплатных вариантов IDE, далее
рассмотрим два варианта: тяжелая суперуниверсальная среда \eclipse, и легкая 
в отношении требуемых ресурсов системы CodeLite.
  
