
  Компилятор преобразует программы на языке программирования в \term{объектный
  код} (смесь кусочков машинного кода со служебной информацией) или в
  текст на языке ассемблера.
  
  \term{Кросс-компилятор}\ (arm-none-eabi-gcc) отличается от обычного
  компилятора тем, что генерирует код не для компьютера на котором он выполняется
  (\term{хост-система}, \verb|$HOST|), а для компьютера другой
  архитектуры\ --- \term{целевой} системы, \verb|$TARGET|.
  
  \term{Ассемблер}\ (as) преобразует человекочитаемый код программы в объектный
  код.
  
  \term{Линкер}\ (ld) объединяет несколько файлов объектного кода в один,
  и корректирует машинный код с учетом его конечного размещения в памяти
  целевой системы (адреса переменных, адреса переходов, размещение сегментов
  кода и данных в физической памяти целевой системы).
  
  \term{Дампер}\ (objdump) преобразует сегменты кода/данных из файла,
  полученного линкером, в формат, необходимый для ПО программатора: бинарные файлы, Intel
  HEX, ELF,.. загружаемые в масочное ПЗУ, FlashPROM (и EEPROM данных на МК
  ATmega).
  
