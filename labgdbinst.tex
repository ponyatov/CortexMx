\labwork{Настрйка отладчика в \eclipse}\label{labgdbinst}

\cp{http://makesystem.net/?p=2146}

 На сегодняшний день существуют много способов и инструментов для отладки
 embedded приложений, начиная с отладки “в железе” (внутрисхемная отладка)  и
 заканчивая всякими симуляторами. У каждого метода есть свои плюсы и минусы, но
 поскольку мы будим писать приложения для реальных устройств, то
 предпочтительней реальная отладка (в железе), то есть приложение будит
 исполняться непосредственно микроконтроллером.
 
Что нам понадобится для “железной отладки” :

\begin{itemize}
  \item 
ARM микроконтроллер (для симуляции необязателен)
  \item 
JTAG/SWD адаптер (для симуляции необязателен)
  \item 
GDB сервер (транслятор интерфейсов GDB/JTAG)
  \item 
GDB отладчик (имеет встроенный симулятор ARM7TDMI, используется для первых лаб) 
  \item 
C/C++ GDB Hardware Debugging plugin
  \item 
Eclipse Embedded Systems Register View plugin
\end{itemize}

JTAG адаптер (он же программатор) следует выбрать тот, который поддерживает
именно ваш микроконтроллер, а еще лучше, микроконтроллеры разных производителей. В моем
случае (еще с давних времен у меня завалялись кристаллы от Texas Instruments, ST
Microelectronics, NXP, Atmel, Cypress), я сразу решил найти программатор
поддерживающий имеющиеся у меня камни. Порыскав в интернетах, мой выбор пал на
китайский клон знаменитого J-Link, в добавок к которому идет уйма полезных
утилит от Segger Microcontroller (тут обошлось без китая \smiley), облегчающие
жизнь разработчику.

В этой книге также рассмотрено несколько простых варинтов JTAG-адаптеров,
которые вы можете сделать сами, не обладая выдающимися знаниями в электронике
и технологиях производства печатных плат.

\bigskip
Структура аппаратно-программного комплекта для отладки:

\menu{микроконтроллер>JTAG/SWD>адаптер>LPT/USB>GDB сервер>протокол
GDB>GDB отладчик>IDE}

Адаптер подключается к выводам МК с помощью колодки (JTAG) или гребенки (SWD).

К компьютеру адаптер подключается через однин из распространенных интерфейсов:
совсем дешевые варинты ``на пяти резисторах'' через порт LPT, чуть подороже
через USB, совсем дорогие проф.модели могут иметь 
