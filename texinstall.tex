\labwork{Установка системы верстки документации \LaTeX}\label{texinstall}

Если вы планируете писать полноценную документацию на программы
и оборудование, или участовать в доделке этой книги, вы можете установить
систему верстки \LaTeX.

Для работы с \TeX\ требуется довольно приличное по усилиям
(само)обучение \cite{lvovsky}, но оно оправдывается если вы часто 
пишете документацию, особенно если в ней больше 10 формул.
Готовить документацию в MS Word\ --- (само)убийство мозга и времени,
идеология подстановочных макросов \TeX, богатый набор доп.пакетов
и командный ввод формул очень доставляют.

\win{Скачайте и установите пакет \miktex:}

\bigskip\wcmd{\url{http://miktex.org/download}>Other Downloads>Net Installer}

\menu{Save as:>\file{D:/ARM/soft/MikTeX/miktex-netsetup-2.9.4503}}

\bigskip Загрузка дистрибутивных файлов

\menu{\file{miktex-netsetup-2.9.4503}>License>Accept>Далее}

\menu{Task>Download>Далее}

Если у вас постоянное internet-соединение: \menu{Package Set>Basic MiKTeX>Далее}

Для offline работы\footnote{когда неизвестно какие пакеты понадобятся\ ---
\miktex\ умеет их докачивать по необходимости} \menu{Package
Set>Complete MikTeX>Далее}

\menu{Download Source>Russian Federation (ctan.uni-altai.ru)>Далее}

\menu{Distribution
Directory>\file{D:/ARM/soft/MikTeX}>Далее>Start>Executing>Далее>Close}

\bigskip Установка из ранее загруженного дистрибутива

\menu{\file{D:/ARM/soft/MikTeX/miktex-\alarm{netsetup}-2.9.4503}>License>Accept>Далее}

\menu{Task>Install>Далее>Basic MiKTeX>Далее}

\menu{Install for>Anyone/Only for user>Далее}

\menu{Install \alarm{from}:>\file{D:/ARM/soft/MikTeX}>Далее}

\menu{Install to:>\file{D:/LaTeX/MiKTeX}>Далее}

\menu{Settings}

\menu{Preferred paper>A4}

Важная опция: автоматическая докачка отсутствующих пакетов
\alarm{\menu{Install missing packages>Yes}}

\menu{Далее>Start>Executing>Close}

\bigskip Двухступенчатая установка позволяет сначала скачать полный дистрибутив
\miktex, а затем установить его на другой компьютер, не подключенный к
\internet, или c медленным/платным каналом не дающим взять и качнуть 200 Мб.

\bigskip Для удобной работы с \file{.tex} файлами в \eclipse\ нужно поставить 
дополнение \file{TeXlipse}:

\bigskip
\menu{\eclipse>Help>Install>Work with>Add}

\menu{Name>TeXlipse}

\menu{Location>\url{http://texlipse.sourceforge.net}}

\dirtree{%}
.1 TeXlipse.
.2 \checkbox\ TeXlipse.
}
